%%
%% This is file `sample-sigconf.tex',
%% generated with the docstrip utility.
%%
%% The original source files were:
%%
%% samples.dtx  (with options: `all,proceedings,bibtex,sigconf')
%% 
%% IMPORTANT NOTICE:
%% 
%% For the copyright see the source file.
%% 
%% Any modified versions of this file must be renamed
%% with new filenames distinct from sample-sigconf.tex.
%% 
%% For distribution of the original source see the terms
%% for copying and modification in the file samples.dtx.
%% 
%% This generated file may be distributed as long as the
%% original source files, as listed above, are part of the
%% same distribution. (The sources need not necessarily be
%% in the same archive or directory.)
%%
%%
%% Commands for TeXCount
%TC:macro \cite [option:text,text]
%TC:macro \citep [option:text,text]
%TC:macro \citet [option:text,text]
%TC:envir table 0 1
%TC:envir table* 0 1
%TC:envir tabular [ignore] word
%TC:envir displaymath 0 word
%TC:envir math 0 word
%TC:envir comment 0 0
%%
%%
%% The first command in your LaTeX source must be the \documentclass
%% command.
%%
%% For submission and review of your manuscript please change the
%% command to \documentclass[manuscript, screen, review]{acmart}.
%%
%% When submitting camera ready or to TAPS, please change the command
%% to \documentclass[sigconf]{acmart} or whichever template is required
%% for your publication.
%%
%%
\documentclass[sigconf,anonymous]{acmart}

\let\comment\undefined

%\usepackage{changes}

%highlight the changes

\usepackage[addedmarkup=colored]{changes}
\makeatletter
\setdeletedmarkup{\@gobble{#1}}
% \setdeletedmarkup{\color{red}\strikeout{#1}}
% \setaddedmarkup{\color{black}#1}
\makeatother
%compile paper with no highlighting
%\usepackage[final]{changes}
%compile paper with no highlighting
 %\usepackage[authormarkup=none,final]{changes}
%\usepackage[final]{changes}
%\usepackage[protrusion=true,expansion=true]{microtype}
%\usepackage{makeidx}

%\usepackage[addedmarkup=uline,deletedmarkup=sout]{changes}
% \usepackage{changes}

% \usepackage{cite}
\usepackage{hyperref}
%Tracking the changes: 

%Fixing the special symbols in the bib files and their corresponding doi for ``splncs04'' bib style. 
\makeatletter
\AtBeginDocument{%
  \def\doi#1{\url{https://doi.org/#1}}}
\makeatother
\usepackage{textcomp} 

%For reducing the free spaces between the sections (This works for reducing the free spaces).
\usepackage{parskip} 
% \usepackage{dblfloatfix}
\usepackage{multicol}
\usepackage{prot}
\usepackage{amsmath}
%\usepackage{breqn}
\let\Bbbk\relax
\usepackage{amssymb,amsfonts}
\usepackage{latexsym}
\usepackage{subcaption}
\usepackage{graphicx}
\usepackage[export]{adjustbox}
\usepackage{fancybox}
\usepackage{floatrow}
\usepackage{tikz}
\setlength{\abovecaptionskip}{-5in} 
\usepackage{tikzit}
\input{1.tikzstyles}
\usepackage{makecell}
\usepackage{lipsum,adjustbox}

\usepackage{pgfplots}
% \usepackage{xintexpr}
%\usepackage[active,tightpage]{preview}
%\PreviewEnvironment{tikzpicture}
\pgfplotsset{compat=newest}
\usetikzlibrary{decorations.pathreplacing}
\usepackage{setspace}
\usepackage{algorithm}
\usepackage[noend]{algpseudocode}
\usepackage[framemethod=tikz]{mdframed}
\usepackage{xspace}
% \usepackage{pgfplots}
\usepackage{framed}
\usepackage{float}
% \pgfplotsset{compat=3}
\usepackage{tabularx}

\setcounter{secnumdepth}{4}

\usepackage{xintexpr}
% \usepackage[active,tightpage]{preview}
\usepackage[tightpage]{preview}

\raggedbottom

\let\llncssubparagraph\subparagraph
%% Provide a definition to \subparagraph to keep titlesec happy
\let\subparagraph\paragraph
% \usepackage{titlesec}
% \titlespacing*{\subsubsection}{0pt}{1ex}{1ex}

\PreviewEnvironment{tikzpicture}
 %\newenvironment{proof}{\noindent{\bf Proof : \ }}{\hfill$\Box$\par\medskip}

%\newenvironment{proof}{\noindent{\bf Proof : \ }}{\hfill$\Box$\par\medskip}

%\newenvironment{construction}{\noindent{}}{}
\newenvironment{proofhack}{\noindent {\bf Proof}.\ }{}
%\newtheorem{theorem}{Theorem}[section]
%\newtheorem{corollary}[theorem]{Corollary}
%\newtheorem{lemma}[theorem]{Lemma}
%\newtheorem{proposition}[theorem]{Proposition}
%\newtheorem{definition}[theorem]{Definition}
%\newtheorem{experiment}[theorem]{Experiment}
%\newtheorem{remark}[theorem]{Remark}
%\newtheorem{claim}[theorem]{Claim}
%\newtheorem{invariant}[theorem]{Invariant}
%\newtheorem{observation}[theorem]{Observation}
%%\newtheorem{example}[theorem]{Example}
%\newtheorem{fact}[theorem]{Fact}
%%\newtheorem{conjecture}[theorem]{Conjecture}
%\newtheorem{problem}[theorem]{Problem}
%\newtheorem{construction}[theorem]{Contruction}
%%\newtheorem{property}[theorem]{Property}

\newenvironment{proofof}[1]{\begin{trivlist} \item {\bf Proof
#1:~~}}
  {\qed\end{trivlist}}
\renewenvironment{proofof}[1]{\par\medskip\noindent{\bf Proof of #1: \ }}{\hfill$\Box$\par\medskip}
\newcommand{\namedref}[2]{\hyperref[#2]{#1~\ref*{#2}}}
\newcommand{\thmlab}[1]{\label{thm:#1}}
\newcommand{\thmref}[1]{\namedref{Theorem}{thm:#1}}
\newcommand{\lemlab}[1]{\label{lem:#1}}
\newcommand{\lemref}[1]{\namedref{Lemma}{lem:#1}}
\newcommand{\claimlab}[1]{\label{claim:#1}}
\newcommand{\claimref}[1]{\namedref{Claim}{claim:#1}}
\newcommand{\corlab}[1]{\label{cor:#1}}
\newcommand{\corref}[1]{\namedref{Corollary}{cor:#1}}
\newcommand{\seclab}[1]{\label{sec:#1}}
\newcommand{\secref}[1]{\namedref{Section}{sec:#1}}
\newcommand{\applab}[1]{\label{app:#1}}
\newcommand{\appref}[1]{\namedref{Appendix}{app:#1}}
\newcommand{\factlab}[1]{\label{fact:#1}}
\newcommand{\factref}[1]{\namedref{Fact}{fact:#1}}
\newcommand{\remlab}[1]{\label{rem:#1}}
\newcommand{\remref}[1]{\namedref{Remark}{rem:#1}}
\newcommand{\figlab}[1]{\label{fig:#1}}
\newcommand{\figref}[1]{\namedref{Figure}{fig:#1}}
\newcommand{\alglab}[1]{\label{alg:#1}}
\renewcommand{\algref}[1]{\namedref{Algorithm}{alg:#1}}
\newcommand{\tablelab}[1]{\label{tab:#1}}
\newcommand{\tableref}[1]{\namedref{Table}{tab:#1}}
\newcommand{\deflab}[1]{\label{def:#1}}
\newcommand{\defref}[1]{\namedref{Definition}{def:#1}}
\newcommand{\explab}[1]{\label{exp:#1}}
\newcommand{\expref}[1]{\namedref{Experiment}{exp:#1}}
\newcommand{\eqnlab}[1]{\label{eq:#1}}
\newcommand{\eqnref}[1]{\namedref{Equation}{eq:#1}}
\newcommand{\constref}[1]{\namedref{\textbf{Construction}}{const:#1}}
\newcommand{\constlab}[1]{\label{const:#1}}
\newcommand{\obslab}[1]{\label{obs:#1}}
\newcommand{\obsref}[1]{\namedref{Observation}{obs:#1}}
\newcommand{\exlab}[1]{\label{ex:#1}}
\newcommand{\exref}[1]{\namedref{Example}{ex:#1}}
\newcommand{\stepref}[1]{\namedref{Step}{step:#1}}
\newcommand{\steplab}[1]{\label{step:#1}}
\newcommand{\propref}[1]{\namedref{Property}{prop:#1}}
\newcommand{\proplab}[1]{\label{prop:#1}}
\newcommand{\probref}[1]{\namedref{Problem}{prob:#1}}
\newcommand{\problab}[1]{\label{prob:#1}}
\newcommand{\conjref}[1]{\namedref{Conjecture}{conj:#1}}
\newcommand{\conjlab}[1]{\label{conj:#1}}
%\newcommand{\propref}[1]{\namedref{Property}{prop:#1}}
%\newcommand{\proplab}[1]{\label{prop:#1}}


\newenvironment{remindertheorem}[1]{\medskip \noindent {\bf Reminder of  #1.  }\em}{}


\def \aadapt    {\mdef{\mathsf{A}_{adaptive}}}
\def \asingle    {\mdef{\mathsf{A}_{single}}}
\def \HAM    {\mdef{\mathsf{HAM}}}
\def \time    {\mdef{\mathsf{time}}}
\def \superconc    {\mdef{\mathsf{superconc}}}
\def \grates    {\mdef{\mathsf{grates}}}
\def \depth    {\mdef{\mathsf{depth}}}
\def \scrypt    {\mdef{\mathsf{scrypt}}}
\def \sinks    {\mdef{\mathsf{sinks}}}
\def \indeg    {\mdef{\mathsf{indeg}}}
\def \parents    {\mdef{\mathsf{parents}}}
\def \ancestors    {\mdef{\mathsf{ancestors}}}
\def \lab    {\mdef{\mathsf{lab}}}
\def \unique    {\mdef{\mathsf{UNIQUE}}}
\def \potential    {\mdef{\mathsf{PotentialParents}}}
\def \partition    {\mdef{\mathsf{BlockPartition}}}
\def \crpartition    {\mdef{\mathsf{CR-BlockPartition}}}
\def \cruniform    {\mdef{\mathsf{CR-Uniform}}}
\def \A    {\mdef{\mathcal{A}}}
\def \D    {\mdef{\mathcal{D}}}
\def \C    {\mdef{\mathcal{C}}}
\def \B    {\mdef{\mathcal{B}}}
\def \M   {\mdef{\mathcal{M}}}
\def \U   {\mdef{\mathcal{U}}}
\def \S   {\mdef{\mathcal{S}}}
\def \R   {\mdef{\mathcal{R}}}
\def \d    {\mdef{\delta}}
\def \q    {\mdef{\mathbf{q}}}
\def \G    {\mdef{\mathbb{G}}}
\def \H    {\mdef{\mathcal{H}}}
\def \MHF    {\mdef{\mathsf{MHF}}}
\def \cmc    {\mdef{\mathsf{cmc}}}
\def \Enc    {\mdef{\mathsf{Enc}}}
\def \Dec    {\mdef{\mathsf{Dec}}}
\def \FHE    {\mdef{\mathsf{FHE}}}

\def \Cond   {\mdef{\mathsf{C}}}
\def \KG   {\mdef{\mathsf{KeyGen}}}
\def \Setup   {\mdef{\mathsf{Setup}}}
\def \Eval   {\mdef{\mathsf{Eval}}}

\def \Prove   {\mdef{\mathsf{Prove}}}
\def \Verify   {\mdef{\mathsf{Verify}}}
\def \Validate   {\mdef{\mathsf{Validate}}}
\def \PKE   {\mdef{\mathsf{PKE}}}
\def \iO {\mdef{i\mathcal{O}}}
\def \GoodKey {\mdef{\mathtt{GOODKEYS}}}


\def \Sim   {\mdef{\mathtt{Sim}}}
\def \CE   {\mdef{\mathtt{CE}}}
\def \AE   {\mdef{\mathtt{AE}}}
\def \PKDF   {\mdef{\mathtt{PKDF}}}
\def \Derive   {\mdef{\mathtt{Derive}}}

\def \lp    {\mdef{\mathsf{lp}}}
\def \pp    {\mdef{\mathsf{pp}}}
\def \LP    {\mdef{\mathsf{LP}}}
\def \bad    {\mdef{\mathsf{bad}}}
\def \chal    {\mdef{\mathsf{chal}}}
\def \sparse    {\mdef{\mathsf{sparse}}}
\def \Setup    {\mdef{\mathsf{Setup}}}
\def \miss    {\mdef{\mathsf{MISS}}}
\def \costly    {\mdef{\mathsf{COSTLY}}}
\def \setup    {\mdef{\textbf{setup}}}
\def \challenge    {\mdef{\textbf{challenge}}}
\def \query    {\mdef{\textbf{query}}}
\def \guess    {\mdef{\textbf{guess}}}
\def \mhfeval  {\mdef{\mathsf{MHF.Eval}}}
\def \etrace    {\mdef{\mathsf{etrace}}}
\def \valiant    {\mdef{\mathsf{Valiant}}}
\def \request    {\mdef{\mathsf{request}}}
\def \store    {\mdef{\mathsf{store}}}
\def \load    {\mdef{\mathsf{load}}}
\def \True    {\mdef{\mathsf{True}}}
\def \False    {\mdef{\mathsf{False}}}
\def \invariant    {\mdef{\mathsf{invariant}}}
\def \permute    {\mdef{\mathsf{permutation}}}
\def \counter    {\mdef{\mathsf{counter}}}
\def \rmerge    {\mdef{\mathsf{RandomMerge}}}
\def \Write    {\mdef{\mathsf{Write}}}
\def \Read    {\mdef{\mathsf{Read}}}
\def \overlay    {\mdef{\mathsf{overlay}}}
\def \SC    {\mdef{\mathsf{SC}}}



\def \Pwd    {\mdef{\mathcal{PWD}}}
\def \Id    {\mdef{\mathcal{ID}}}
\def \pwd   {\mdef{\mathtt{pwd}}}
\def \salt   {\mdef{\mathtt{salt}}}
\def \msg   {\mdef{\mathsf{msg}}}
\def \id   {\mdef{\mathtt{Id}}}



%\def \RNU    {\mdef{\mathsf{RegisterNewUser}}}

\def \QSgin    {\mdef{\mathsf{RegisterQuery}}}

\def \Lgin    {\mdef{\mathsf{Login}}}
\def \QLgin    {\mdef{\mathsf{LoginQuery}}}
\def \Updt    {\mdef{\mathsf{Update}}}
\def \Act    {\mdef{\mathsf{Act}}}
\def \PwdUp    {\mdef{\mathsf{PwdUpdate}}}
\def \PBE    {\mdef{\mathsf{PBE}}}
\def \SE    {\mdef{\mathsf{SE}}}
\def \SH    {\mdef{\mathsf{SH}}}
\def \PCache    {\mdef{\mathsf{PCache}}}
\def \SDis  {\mdef{\mathsf{SmallDistance}}}
\def \TpDis  {\mdef{\mathsf{tpDist}}}
\def \TgDis  {\mdef{\mathsf{tgtDist}}}
\def \Hamm  {\mdef{\mathsf{Hamming}}}
\def \Edit  {\mdef{\mathsf{Edit}}}
\def \Dist  {\mdef{\mathtt{dist}}}

\def \Del {\mdef{\mathtt{Delete}}}
\def \Ins {\mdef{\mathtt{Insert}}}
\def \Chng {\mdef{\mathtt{Change}}}
\def \Trasn {\mdef{\mathtt{Transfer}}}

\def \Extr  {\mdef{\mathtt{Extract}}}
\def \EqPwdVec {\mdef{\mathtt{EqPwdVec}}}
\def \ValidLginQ {\mdef{\mathtt{ValidLginQuery}}}
\def \ToInt  {\mdef{\mathsf{ToInt}}}
\def \ToOrig  {\mdef{\mathsf{ToInt}^{-1}}}
\def \Pad  {\mdef{\mathtt{Pad}}}
\def \UPad  {\mdef{\mathtt{UnPad}}}
\def \Pail {\mdef{\mathtt{P}}}
\def \SS {\mdef{\mathtt{SS}}}
\def \PtoCMul {\mdef{\mathtt{PlainToCtxMul}}}
\def \Add {\mdef{\mathtt{Add}}}
\def \Sum {\mdef{\Sigma}}
\def \Sub {\mdef{\mathtt{Subtract}}}
\def \lcm {\mdef{\mathsf{lcm}}}
\def \Zs {\mdef{\mathbb{Z}^*}}
\newcommand\Mbb[1]{\mathbb{#1}}
%predicates macros
\def \HammOne {\mdef{\mathtt{HammDisOne}}}
\def \ArbHamm {\mdef{\mathtt{HammDis}}}
\def \Capslock {\mdef{\mathsf{CAPSLOCK}}}
\def \oneDel {\mdef{\mathtt{OneCharDel}}}
\def \oneIns {\mdef{\mathtt{OneCharIns}}}
\def \Sum   {\mdef{\mathtt{Sum}}}
\def \RanEnc   {\mdef{\mathsf{REnc}}}
\def \RanDec   {\mdef{\mathsf{RDec}}}

\def \Auth   {\mdef{\mathsf{Auth}}}


\def \Ham  {\mdef{\mathtt{Ham}}}
\def \ED  {\mdef{\mathtt{ED}}}
\def \ToLowerCase  {\mdef{\mathtt{ToLowerCase}}}

\def \apnd  {\mdef{\mathtt{Append}}}

\def \BboxP  {\mdef{\mbox{\fontsize{13.28}{21.6}\selectfont\( %
			\boxplus%
			\)} }}




\def \Init {\mdef{\mathsf{Initialization}}}
\def \Sgin {\mdef{\mathsf{RegisterNewUser}}}
\def \Lgin {\mdef{\mathsf{Login}}}
\def \EvType {\mdef{\mathsf{EventType}}}
\def \Register {\mdef{\mathsf{Register}}}
\def \Login {\mdef{\mathsf{Login}}}
\def \PwdUpd {\mdef{\mathsf{PwdUpdate}}}
\def \acpt {\mdef{\mathtt{acpt}}}
\def \rjct {\mdef{\mathtt{rjct}}}
\def \isReg {\mdef{\mathtt{isRegistered}}}
\def \PullPwd {\mdef{\mathtt{PullPwd}}}
\def \view {\mdef{\mathtt{view}}}

\def \Share   {\mdef{\mathsf{ShareGen}}}
\def \recover   {\mdef{\mathsf{SecretRecover}}}
\def \ValidShr   {\mdef{\mathsf{ValidShare}}}
\def \InterPol   {\mdef{\mathsf{InterPol}}}
\def \shrs   {\mdef{\mathsf{shares}}}
\def \Diffx   {\mdef{\mathtt{Diff}}}
\def \Experiment   {\mdef{\mathtt{Experiment}}}
\def \Equ   {\mdef{\mathtt{Equ}}}

\def \badRng   {\mdef{\mathtt{BadRange}}}

\newcommand{\InvertCase}{\mathsf{InvertCase}}

\def \iif   {\mdef{\mathtt{if~}}}
\def \eelse   {\mdef{\mathtt{else~}}}
\def \ffor   {\mdef{\mathtt{for~}}}
\def \bbreak   {\mdef{\mathtt{break~}}}
\def \ooutput   {\mdef{\mathtt{output~}}}
\def \ddiscard   {\mdef{\mathtt{discard~}}}
\def \ccompute   {\mdef{\mathtt{compute~}}}
\def \sset   {\mdef{\mathtt{set~}}}
\def \ssample   {\mdef{\mathtt{sample~}}}
\def \pparse   {\mdef{\mathtt{parse~}}}
\def \aand   {\mdef{\mathtt{~AND~}}}
\def \oor   {\mdef{\mathtt{~OR~}}}
\def \run   {\mdef{\mathtt{run~}}}
\def \rreturn   {\mdef{\mathtt{return~}}}
\def \condit   {\mdef{\mathtt{Cond}}}

\newcommand{\ldb}{\mathopen{\lbrack\!\lbrack}} 
\newcommand{\rdb}{\mathclose{\rbrack\!\rbrack}}




\renewcommand{\algorithmiccomment}[1]{\hfill$\triangleright${\color{purple} #1}}
\renewcommand{\algorithmicrequire}{\textbf{Input:}}
\renewcommand{\algorithmicensure}{\textbf{Output:}}
%---------------------------------------------------------------------------------------------------------------------------------------------------

%---------------------------------------------------------------------------------------------------------------------------------------------------

\newcommand\norm[1]{\left\lVert#1\right\rVert}
\newcommand{\PPr}[1]{\ensuremath{\mathbf{Pr}\left[#1\right]}}
\newcommand{\PPPr}[2]{\ensuremath{\underset{#1}{\mathbf{Pr}}\left[#2\right]}}
\newcommand{\Ex}[1]{\ensuremath{\mathbb{E}\left[#1\right]}}
\newcommand{\EEx}[2]{\ensuremath{\underset{#1}{\mathbb{E}}\left[#2\right]}}
\renewcommand{\O}[1]{\ensuremath{\mathcal{O}\left(#1\right)}}
\newcommand{\eps}{\epsilon}

\renewcommand{\labelenumi}{(\arabic{enumi})}
%\newcommand{\construction}[1]{\textbf{Construction} #1}
\renewcommand\labelitemii{\mbox{$\circ$}}
\renewcommand\labelitemiii{\mbox{\tiny$\spadesuit$}}
\renewcommand{\figurename}{Figure}

%%% Unified Formating
\newcommand{\mdef}[1]{{\ensuremath{#1}}\xspace}  % Math Def which can also be used in normal text.
\newcommand{\mydist}[1]{\mdef{\mathcal{#1}}}     % Distribution should use mathcal.
\newcommand{\myset}[1]{\mdef{\mathbb{#1}}}       % (Important) Sets should use mathbb.
\newcommand{\myalg}[1]{\mdef{\mathcal{#1}}}       % Algorithms should use mathtt.
\newcommand{\myfunc}[1]{\mdef{\mathsf{#1}}}      % Functions denoted with text (e.g. size()) should use mathsf.
\newcommand{\myvec}[1]{\mathbf{#1}}               % Vector
\def \ds {\displaystyle}                         % Shorthand for ``displaystyle''.
\DeclareMathOperator*{\argmin}{argmin}
\DeclareMathOperator*{\argmax}{argmax}
\DeclareMathOperator*{\polylog}{polylog}
\DeclareMathOperator*{\poly}{poly}

%%% Text notation
\newcommand{\superscript}[1]{\ensuremath{^{\mbox{\tiny{\textit{#1}}}}}\xspace}
\def \th {\superscript{th}}     % 'The i-th entry it a list...' --> i\th
\def \st {\superscript{st}}     % 'The 1-st entry it a list...' --> 1\st
\def \nd {\superscript{nd}}     % 'The 2-nd entry it a list...' --> 2\nd
\def \rd {\superscript{rd}}     % 'The 3-rd entry it a list...' --> 3\rd
\def \etal{{\it et~al.}}

%%% Common Sets & Symbols & Functions
\def \size     {\mdef{\myfunc{size}}}                % Function outputing vector of cardinalities of a given vector of sets.
\def \negl     {\mdef{\myfunc{negl}}}                % Negliglbe function
%\def \polylog  {\mdef{\myfunc{polylog}}}             % Polylogarithm function
\def \polyloglog  {\mdef{\myfunc{polyloglog}}}       % Any function that is polynomial in the logarithm of the logarithm of the argument
\newcommand{\abs}[1]{\mdef{\left|#1\right|}}         % Absolute value
\newcommand{\flr}[1]{\mdef{\left\lfloor#1\right\rfloor}}              % Absolute value
\newcommand{\ceil}[1]{\mdef{\left\lceil#1\right\rceil}}               % Absolute value
\newcommand{\rdim}[1]{\mdef{\dim\left(#1\right)}}                     % Dimension of a vector
\newcommand{\set}[1]{\mdef{\left\{#1\right\}}}                        % Absolute value
\newcommand{\E}[2][]{\mdef{\underset{#1}{\mathbb{E}}\left[#2\right]}} % Expected value

\newcommand{\ignore}[1]{}


%%%% pebbling complexity notions
\def \cc       {\mdef{\mathsf{cc}}}
\newcommand{\peb}{\Pi} % pebbling complexity
\newcommand{\Peb}{{\cal P}} % set of legal pebblings

%\newcommand{\parallel}{\|}

\newcommand{\ppeb}{\peb^{\parallel}} % parallel versions of def above
\newcommand{\pPeb}{\Peb^{\parallel}}

\newcommand{\Cspace}{s}
\newcommand{\Ctime}{t}
\newcommand{\Ccc}{{cc}}
\newcommand{\Cspacetime}{{st}}
\newcommand{\Cevery}{\{\Cspace,\Ctime,\Cspacetime,\Ccc\}}

%\def \pcc {\ppeb_\Ccc} % shortcut for Parallel Cumulative Complexity
\def \pcc {\cc} % shortcut for Parallel Cumulative Complexity
\def \sst {\peb_{\Cspacetime}} % shourtcut for Sequential Space Time


\newif\ifnotes\notestrue %set this to true if notes are visible and to false (next line) if they should be hidden
% \newif\ifnotes\notesfalse
\ifnotes
\newcommand{\samson}[1]{\textcolor{purple}{{\bf (Samson:} {#1}{\bf ) }} \marginpar{\tiny\bf
             \begin{minipage}[t]{0.5in}
               \raggedright S:
            \end{minipage}}}       
            
\newcommand{\jeremiah}[1]{\textcolor{red}{{\bf (Jeremiah:} {#1}{\bf ) }} \marginpar{\tiny\bf
             \begin{minipage}[t]{0.5in}
               \raggedright J:
            \end{minipage}}    }         
        
\newcommand{\hassan}[1]{\textcolor{blue}{{\bf (Hassan:} {#1}{\bf ) }} \marginpar{\tiny\bf
		\begin{minipage}[t]{0.5in}
			\raggedright H:
\end{minipage}}    }
                							
\else
\newcommand{\samson}[1]{}
\newcommand{\jeremiah}[1]{}
\fi

\hypersetup{
     colorlinks   = true,
     citecolor    = blue,
		 linkcolor		= red
}

\makeatletter
\renewcommand*{\@fnsymbol}[1]{\textcolor{mahogany}{\ensuremath{\ifcase#1\or *\or \dagger\or \ddagger\or
 \mathsection\or \triangledown\or \mathparagraph\or \|\or **\or \dagger\dagger
   \or \ddagger\ddagger \else\@ctrerr\fi}}}
\makeatother

\providecommand{\email}[1]{\href{mailto:#1}{\nolinkurl{#1}\xspace}}

\definecolor{electricpurple}{rgb}{0.75, 0.0, 1.0}
\definecolor{fluorescentpink}{rgb}{1.0, 0.08, 0.58}
\definecolor{mahogany}{rgb}{0.75, 0.25, 0.0}
\definecolor{darkblue}{rgb}{0.0, 0.0, 0.55}
\definecolor{darkpastelgreen}{rgb}{0.01, 0.75, 0.24}
\definecolor{darkgreen}{rgb}{0.0, 0.2, 0.13}
\definecolor{darkgoldenrod}{rgb}{0.72, 0.53, 0.04}
\definecolor{darkred}{rgb}{0.55, 0.0, 0.0}
\hypersetup{
     colorlinks   = true,
     citecolor    = electricpurple,
		 linkcolor		= blue
}
\fancyhf{none}
\setcopyright{none}
\acmYear{2024}
\settopmatter{printacmref=false, printccs=true, printfolios=true} 
\usepackage[subtle]{savetrees}
\pagestyle{plain}

%


%% Rights management information.  This information is sent to you
%% when you complete the rights form.  These commands have SAMPLE
%% values in them; it is your responsibility as an author to replace
%% the commands and values with those provided to you when you
%% complete the rights form.
%\setcopyright{acmlicensed}
%\copyrightyear{2018}
%\acmYear{2018}
%\acmDOI{XXXXXXX.XXXXXXX}

%% These commands are for a PROCEEDINGS abstract or paper.
%\acmConference[Conference acronym 'XX]{Make sure to enter the correct
%  conference title from your rights confirmation emai}{June 03--05,
 % 2018}{Woodstock, NY}
%%
%%  Uncomment \acmBooktitle if the title of the proceedings is different
%%  from ``Proceedings of ...''!
%%
%%\acmBooktitle{Woodstock '18: ACM Symposium on Neural Gaze Detection,
%%  June 03--05, 2018, Woodstock, NY}
%\acmISBN{978-1-4503-XXXX-X/18/06}


%%
%% Submission ID.
%% Use this when submitting an article to a sponsored event. You'll
%% receive a unique submission ID from the organizers
%% of the event, and this ID should be used as the parameter to this command.
%%\acmSubmissionID{123-A56-BU3}

%%
%% For managing citations, it is recommended to use bibliography
%% files in BibTeX format.
%%
%% You can then either use BibTeX with the ACM-Reference-Format style,
%% or BibLaTeX with the acmnumeric or acmauthoryear sytles, that include
%% support for advanced citation of software artefact from the
%% biblatex-software package, also separately available on CTAN.
%%
%% Look at the sample-*-biblatex.tex files for templates showcasing
%% the biblatex styles.
%%

%%
%% The majority of ACM publications use numbered citations and
%% references.  The command \citestyle{authoryear} switches to the
%% "author year" style.
%%
%% If you are preparing content for an event
%% sponsored by ACM SIGGRAPH, you must use the "author year" style of
%% citations and references.
%% Uncommenting
%% the next command will enable that style.
%%\citestyle{acmauthoryear}


%%
%% end of the preamble, start of the body of the document source.
\begin{document}
%%
%% The "title" command has an optional parameter,
%% allowing the author to define a "short title" to be used in page headers.
\title{Conditional Encryption with Applications to Secure Personalized Password Typo Correction}

%%
%% The "author" command and its associated commands are used to define
%% the authors and their affiliations.
%% Of note is the shared affiliation of the first two authors, and the
%% "authornote" and "authornotemark" commands
%% used to denote shared contribution to the research.
%\author{Anonymous authors}




%%
%% By default, the full list of authors will be used in the page
%% headers. Often, this list is too long, and will overlap
%% other information printed in the page headers. This command allows
%% the author to define a more concise list
%% of authors' names for this purpose.

%%
%% The abstract is a short summary of the work to be presented in the
%% article.
\begin{abstract}
\input{abstract}
\end{abstract}




%%
%% The code below is generated by the tool at http://dl.acm.org/ccs.cfm.
%% Please copy and paste the code instead of the example below.
%%

\begin{CCSXML}
<ccs2012>
   <concept>
       <concept_id>10002978.10003029.10011703</concept_id>
       <concept_desc>Security and privacy~Usability in security and privacy</concept_desc>
       <concept_significance>500</concept_significance>
       </concept>
   <concept>
       <concept_id>10002978.10002991.10002992</concept_id>
       <concept_desc>Security and privacy~Authentication</concept_desc>
       <concept_significance>500</concept_significance>
       </concept>
   <concept>
       <concept_id>10002950.10003648.10003702</concept_id>
       <concept_desc>Mathematics of computing~Nonparametric statistics</concept_desc>
       <concept_significance>500</concept_significance>
       </concept>
   <concept>
       <concept_id>10002978.10002991.10002992</concept_id>
       <concept_desc>Security and privacy~Authentication</concept_desc>
       <concept_significance>500</concept_significance>
       </concept>
   <concept>
       <concept_id>10002978.10002979.10002981.10011745</concept_id>
       <concept_desc>Security and privacy~Public key encryption</concept_desc>
       <concept_significance>500</concept_significance>
       </concept>
   <concept>
       <concept_id>10002978.10002991.10002992</concept_id>
       <concept_desc>Security and privacy~Authentication</concept_desc>
       <concept_significance>500</concept_significance>
       </concept>
   <concept>
       <concept_id>10002978.10003029.10011703</concept_id>
       <concept_desc>Security and privacy~Usability in security and privacy</concept_desc>
       <concept_significance>300</concept_significance>
       </concept>
 </ccs2012>
\end{CCSXML}

\ccsdesc[500]{Security and privacy~Usability in security and privacy}
\ccsdesc[500]{Security and privacy~Authentication}
\ccsdesc[500]{Mathematics of computing~Nonparametric statistics}
\ccsdesc[500]{Security and privacy~Authentication}
\ccsdesc[500]{Security and privacy~Public key encryption}
\ccsdesc[500]{Security and privacy~Authentication}
\ccsdesc[300]{Security and privacy~Usability in security and privacy}

%%
%% Keywords. The author(s) should pick words that accurately describe
%% the work being presented. Separate the keywords with commas.
\keywords{Conditional Encryption, Password Typos, Public Key Encryption}

\maketitle
\input{Intro}
% !TEX root = main.tex

\subsection{Preliminaries}\seclab{preliminaries}
In this section, we review the notations and cryptographic primitive which will be used in the rest of the paper. \newline 

Given a randomized algorithm $\A$ (e.g., key-generation)  we use $y=A(x;r)$ to denote the deterministic output of $A$ when run on input $x$ with fixed random $r \in \{0,1\}^*$ and we use the random variable  $y \leftarrow \A(x)$ to denote the output of $A(x;r)$ when $r$ is selected randomly.  

Let  $ \Sigma $ denote an alphabet (e.g., ASCII or unicode). Given a string $w \in \Sigma^*$ we use $\norm{w}$ to denote the length of $w$ and for $i \leq \norm{w}$ we use $w[i]$ to denote the $i$th character of $w$. We let $\M_n = \Sigma^{\leq n}$ denote the set of all strings $w$ with length $\norm{w} \leq n$. It will be convenient to assume that all passwords have the same length. Of course most user passwords do not have the same length but if the maximum length of a user password is $n-1$ then we can easily define a $1$ to $1$ function $\Pad:\Sigma^{\leq n-1} \rightarrow \Sigma^n$ and consider $\Pwd = \Sigma^n$ to be the set of all possible user passwords after padding. In practice, we could select $n=30$ as essentially all user passwords are shorter than this (e.g., over $99.9\%$ of leaked RockYou passwords were less than 30 characters.). The symbol ``$\|$'' will be used for concatenation. Thus, $ y = x_1 \| x_2 $ is concatenation of $ x_1 $ and $ x_2 $. \newline

Let $ L = \langle l_1, \ldots, l_{|L|}  \rangle $ be list of $ |L| $ elements. We also define the operation $ L' = \apnd(L, l) $ which adds $ l $ to the list and we have $ L' =  \langle l_1, \ldots, l_{|L|}, l \rangle $. We note that $ l_i $ can be an element in $ \Mbb{Z}_{N^2}, \Sigma^n, \M_n$ or $ \Pwd $, etc.  \newline

For the message $m \in \Sigma^{\leq n}$ we use the notation  $m_{-i} \in \Sigma^{\leq n-1} $ to denote the string $ m$ when the $ i $-th char is deleted and if $i > |m|$ then $m_{-i}=m$.

\subsubsection{Partially homomorphic Encryption}
The  Paillier cryptosystem is a partially homomorphic cryptographic scheme which supports ciphertext addition, plaintext to ciphertext multiplication and subtraction.  Specifically, the public key $pk=(N,g)$ (resp. secret key $sk=(\beta, \mu)$) consists of $N=pq$ where $p,q$ are prime numbers and the number $g=N+1 \in \mathbb{Z}_{N^2}^*$ (resp. $\beta =\lcm(p-1,q-1)$ and $\mu=\varphi(N)^{-1} \mod N$). We note that for all $i \in \mathbb{Z}_N$ we have $g^i = \sum_{j=0}^i {i \choose j} N^j = 1+Ni \mod{N^2}$ so that $g$ has multiplicative order $N$ modulo $N^2$. The secret key $sk=(\beta,\mu)$ consists of two parameters $\beta=\lcm(p-1,q-1)$ and $\mu = \varphi(N)^{-1} \mod N$ is defined to be the multiplicative inverse of $\varphi(N) = (p-1)(q-1)$ modulo $N$. \\ 

The algorithm $\Enc_{pk}(m;r)$ takes as input a message $m \in \mathbb{Z}_N$ and a nonce $r \in \mathbb{Z}_N^*$ and outputs $g^{m}r^N \mod{N^2}$. The function $\Enc_{pk}$ acts as a bijective map from $\mathbb{Z}_N \times \mathbb{Z}_N^* \rightarrow \mathbb{Z}_{N^2}^*$. In particular, for {\em every} $c \in \mathbb{Z}_{N^2}^*$ there is a message $m \in \mathbb{Z}_N$ and a nonce $r \in \mathbb{Z}_N^*$ such that $c=g^m r^N \mod{N^2}$ \cite{EC:Paillier99}. \\

%The key generation algorithm is exactly the same as the original key generation algorithm. However, the only difference is that we repeat the  key generation algorithm until to make sure for all possible messages  $ m\in \M$ we have $  m < \frac{\sqrt{N}}{100}$. So basically, we can make sure that all the messages are co-prime to the chose secret keys.  \hassan{I wanted to make sure that this way of generating the secret keys are ok or not}
%\hassan{We may need to omit this part as we will discuss it when we present the construction.}

The encryption scheme has several homomorphic properties in particular if $c_1 = g^{m_1} r_1^N \mod{N^2}$ and $c_2= g^{m_2} r_2^N \mod{N^2}$ encrypt message $m_1,m_2 \in \mathbb{Z}_N$ respectively then $c_1c_2 = g^{m_1+m_2} (r_1r_2)^N \mod{N^2}$ encrypts the message $m_1+m_2 \mod N$. Similarly, if $c= g^m r^N \mod {N^2}$ encrypts the message $m$ then $c^k = g^{mk} (r^k)^N \mod {N^2}$ encrypts the message $mk \mod N$. See \appref{App:PailDetails} for a full description of the Paillier encryption scheme. \\

%\hassan{We can ignore this part in this paper. Just double check to see if we used this notation in the rest of the paper or not.} For convenience we define operations $\boxplus$ and $\boxminus$ where $c_1 \boxplus c_2 \doteq c_1 c_2 \mod{N^2}$ and $c_1 \boxminus c_2 \doteq c_1 c_2^{-1} \mod{N^2}$. Intuitively, if $c_1,c_2 \in \mathbb{Z}_{N^2}^*$ correspond to messages $m_1,m_2 \in \mathbb{Z}_N$ then the ciphertext $c=c_1 \boxplus c_2 \in \mathbb{Z}_{N^2}^*$ (resp. $c=c_1 \boxminus c_2 \in \mathbb{Z}_{N^2}$) encrypts the message $m_1+m_2 \mod N$ (resp. $m_1-m_2 \mod{N}$). Similarly, given $k \in \mathbb{Z}_N$ and ciphertext $c$ encrypting $m \in \mathbb{Z}_N$ we define $k \boxtimes c \doteq c^k \mod{N^2}$ which corresponds to the message $mk \mod{N}$. Finally, given ciphertexts $c_1,\ldots, c_k$ corresponding to messages $m_1,\ldots, m_k $  will be convenient to define $\BboxP_{i=1}^k c_i \doteq \prod_{i=1}^k c_i \mod{N^2}$  which corresponds to the message $\sum_{i=1}^k m_i \mod{N}$. \\


When we apply the Paillier Cryptosystem, our desired message space $\mathcal{M}$ is typically not the set of integers $\mathbb{Z}_N$. Thus, we assume that there is an \replaced{injective}{invective} map $ \ToInt: \mathcal{M} \to \Mbb{N} $ and $ \ToOrig: \Mbb{N} \to  \mathcal{M}$. We will also assume that $|\mathcal{M}| \leq N$ and that $\forall m \in \mathcal{M}$ that $0 \leq \ToInt(m) < \left| \mathcal{M}\right| \leq  N$. Given $x \in \mathbb{Z}_N$ we define $\ToOrig (x) = \bot$ if $x$ has no preimage i.e., $\forall m \in \mathcal{M}$ we have $\ToInt(m) \neq x$.
\subsubsection{Secret Sharing (SS)}\seclab{Sec:SecretSharing}
Several of our constructions rely on a primitive called secret sharing. A $(t,n)$-secret sharing scheme consists of two polynomial time algorithms $\mathtt{ShareGen}$ and $\recover$. Intuitively, $\left( \ldb s \rdb_1, \ldots,  \ldb s \rdb_n\right) \gets \mathtt{ShareGen}(n,t,s)$ takes as input a secret $s \in \mathbb{F}$ along with parameters $n,t$ and outputs $n$ shares $\left(\ldb s \rdb_1, \ldots, \ldb s \rdb_n\right) \in \mathbb{F}$. Given any subset $S = \{i_1,\ldots, i_t\} \subseteq [n]$ of $|S| = t$ shares we can recover the secret $s$ using  $$\recover\left(\left(i_1, \ldb s \rdb_{i_1}), \ldots, (i_t, \ldb s \rdb_{i_t}\right)\right) = s$$ However, given any smaller subset $S=\{i_1,\ldots, i_{t-1}\} \subseteq [n]$ of size $|S| \leq t-1$ shares an attacker cannot infer {\em anything} about $s$ from the shares $\ldb s \rdb_{i_1}, \ldots, \ldb s \rdb_{i_{t-1}}$. In particular, we require that  for all secrets $s \in \mathbb{F}$,  all subsets $S = \{i_1,\ldots, i_{t-1}\} \subseteq [n]$ of size $t-1$ the shares  $\ldb s \rdb_{i_1}, \ldots,  \ldb s \rdb_{i_{t-1}}$ can be viewed as uniformly random independent elements in $\mathbb{F}$ unrelated to the secret $s$. The Shamir Secret sharing scheme \cite{CACM:Shamir79} satisfies this requirement. See appendix \appref{Sec:SecretSharing} for more detail about (Shamir) Secret Sharing.

\subsubsection{String Distance and Close Passwords} Given a string $w \in \Sigma^n$ and $i \leq n$ we use $w[i] \in \Sigma$ to denote the $i$th character of $\Sigma$ and given two strings $w_1,w_2 \in \Sigma^n$ we use $\Ham(w_1,w_2) = \left|\{ i | w[i] \neq w[j]\} \right|$ to denote the hamming distance between them. Similarly, given two strings $w_1, w_2 \in \Sigma^*$ we use $\ED(w_1,w_2)$ to denote the edit-distance between them i.e., the minimum number of insertions/deletions to transform $w_1$ into $w_2$ (or vice versa). Note that if $w_1=w_2$ then $\Ham(w_1,w_2)=0$ and $\
\ED(w_1,w_2)=0$. We will often use Hamming/Edit Distance to determine if two passwords $\pwd_1,\pwd_2$ are close e.g., we could define a predicate $P(\pwd_1,\pwd_2)=1$ if $\Ham(\pwd_1, \pwd_2) \leq 2$ or $\ED(\pwd_1, \pwd_2) \leq 1$; otherwise $P(\pwd_1,\pwd_2)=0$. We could also combine Hamming/Edit distance with other common password typos such as CAPSLOCK/SHIFT errors e.g., $P(\pwd_1, \pwd_2)=1$ if $\InvertCase(\pwd_1) = \pwd_2$ or $\Ham(\pwd_1, \pwd_2) \leq 2$ or $\ED(\pwd_1, \pwd_2) \leq 1$; otherwise, $P(\pwd_1,\pwd_2)=0$.
% \input{TechnicalOverview}
\input{conditionalencryption.tex}

%!TEX root = main.tex

 %
\begin{figure*}
\begin{center}
\captionsetup{justification=raggedright}
\ffigbox[.8\textwidth]{%
\begin{subfloatrow}[3]
\hspace{-.15\textwidth}\ffigbox[\FBwidth]{\caption{CDec Time for HamDist}
		\label{CondDecHD}
		\figlab{fig:HDdifT}}
  {
  \resizebox{6cm}{!}{\begin{tikzpicture}
		\begin{axis}[
        tick label style={font=\fontsize{30}{30}\selectfont},
        width=6cm,
        xmin = 0, xmax = 140,
		ymin = 0, ymax = 100000000000,
		ymode=log,
		xtick distance = 32,
		ytick distance = 1000,
		xlabel =  {\fontsize{30}{30}\selectfont Input Length (n)},
		ylabel = {\fontsize{30}{30}\selectfont Time  (msec)},
		grid = both,
		minor tick num = 1,
		major grid style = {lightgray},
		minor grid style = {lightgray!25},
		width = 1.1\textwidth,
		height = .9\textwidth,
		%legend cell align = {left},
		legend style={at={(0.1,-0.4)},anchor=north}
		%legend pos = north west
		]
		
		%\pgfplotstableread{../../test/CondEnc/HDdataL8_T.dat}{\table}
		%\addplot[red, mark = *] table [x = {T}, y = {CondDec}] {\table};
		
		\pgfplotstableread{../test/CondEnc/HDdataL_T1.dat}{\table}
		\addplot[red, mark = square*] table [x = {L}, y = {CondDec}] {\table};
		\label{T1}
  
        \pgfplotstableread{../test/CondEnc/OPT_HDdataL_T1.dat}{\table}
		\addplot[blue, mark = square*] table [x = {L}, y = {CondDec}] {\table};
		\label{OPTT1}
		
		\pgfplotstableread{../test/CondEnc/HDdataL_T2.dat}{\table}
		\addplot[red, mark = triangle*] table [x = {L}, y = {CondDec}] {\table};
		\label{T2}

        \pgfplotstableread{../test/CondEnc/OPT_HDdataL_T2.dat}{\table}
		\addplot[blue, mark = triangle*] table [x = {L}, y = {CondDec}] {\table};
		\label{OPTT2}
		
		\pgfplotstableread{../test/CondEnc/HDdataL_T3.dat}{\table}
		\addplot[red, mark = *] table [x = {L}, y = {CondDec}] {\table};
		\label{T3}

        \pgfplotstableread{../test/CondEnc/OPT_HDdataL_T3.dat}{\table}
		\addplot[blue, mark = *] table [x = {L}, y = {CondDec}] {\table};
		\label{OPTT3}
		
		\pgfplotstableread{../test/CondEnc/HDdataL_T4.dat}{\table}
		\addplot[red, mark = square*,  mark options={fill=white} ] table [x = {L}, y = {CondDec}] {\table};
		\label{T4}

        \pgfplotstableread{../test/CondEnc/OPT_HDdataL_T4.dat}{\table}
		\addplot[blue, mark = square*,  mark options={fill=white} ] table [x = {L}, y = {CondDec}] {\table};
		\label{OPTT4}

		\end{axis}
		
			\node [draw,fill=white] at (rel axis cs: .25,.8) {\shortstack[l]{
				  {\fontsize{30}{30}\selectfont $\ell = 1$ \ref{T1}, $\ell = 1$ \ref{OPTT1}}   \\
				{\fontsize{30}{30}\selectfont $\ell = 2$ \ref{T2}, $\ell = 2$ \ref{OPTT2}}  \\
				{\fontsize{30}{30}\selectfont $\ell = 3$ \ref{T3}, $\ell = 3$ \ref{OPTT3} } \\
				{\fontsize{30}{30}\selectfont  $\ell =4$ \ref{T4}, $\ell = 4$ \ref{OPTT4}} \\
    {\fontsize{30}{30} \selectfont {\color{red} orig}, \fontsize{30}{30} \selectfont {\color{blue} optimized} } } };
		
		\end{tikzpicture}}}
\ffigbox[\FBwidth]{\caption{CDec Time for HamDist}\figlab{fig:HDdifL}\label{CondDecHD_T}}
  {
  \resizebox{6cm}{!}{\begin{tikzpicture}
	\begin{axis}[tick label style={font=\fontsize{30}{30}\selectfont},
	xmin = 0, xmax = 5,
	ymin = 0, ymax = 10000000000,
	ymode=log,
	xtick distance = 1,
	ytick distance = 1000,
	xlabel = {\fontsize{30}{30}\selectfont Max Haming Distance},
	ylabel ={\fontsize{30}{30}\selectfont Time  (msec)},
	grid = both,
	minor tick num = 1,
	major grid style = {lightgray},
	minor grid style = {lightgray!25},
	width = 1.1\textwidth,
	height = .9\textwidth,
	%legend cell align = {left},
	legend style={at={(0.1,-0.25)},anchor=north}
	%legend pos = north west
	]
	
	\pgfplotstableread{../test/CondEnc/HDdataL8_T.dat}{\table}
	\addplot[red, mark = *] table [x = {T}, y = {CondDec}] {\table};
	\label{L8}

    \pgfplotstableread{../test/CondEnc/OPT_HDdataL8_T.dat}{\table}
	\addplot[blue, mark = *] table [x = {T}, y = {CondDec}] {\table};
	\label{OPTL8}
	
	\pgfplotstableread{../test/CondEnc/HDdataL16_T.dat}{\table}
	\addplot[red, mark = square*] table [x = {T}, y = {CondDec}] {\table};
	\label{L16}
 
 \pgfplotstableread{../test/CondEnc/OPT_HDdataL16_T.dat}{\table}
	\addplot[blue, mark = square*] table [x = {T}, y = {CondDec}] {\table};
	\label{OPTL16}
	
	\pgfplotstableread{../test/CondEnc/HDdataL32_T.dat}{\table}
	\addplot[red, mark = triangle*] table [x = {T}, y = {CondDec}] {\table};
	\label{L32}

    \pgfplotstableread{../test/CondEnc/OPT_HDdataL32_T.dat}{\table}
	\addplot[blue, mark = triangle*] table [x = {T}, y = {CondDec}] {\table};
	\label{OPTL32}
 
	
	\pgfplotstableread{../test/CondEnc/HDdataL64_T.dat}{\table}
	\addplot[red, mark = *] table [x = {T}, y = {CondDec}] {\table};
	\label{L64}

    \pgfplotstableread{../test/CondEnc/OPT_HDdataL64_T.dat}{\table}
	\addplot[blue, mark = *] table [x = {T}, y = {CondDec}] {\table};
	\label{OPTL64}
	
	\pgfplotstableread{../test/CondEnc/HDdataL128_T.dat}{\table}
	\addplot[red, mark = square*,  mark options={fill=white} ] table [x = {T}, y = {CondDec}] {\table};
	\label{L128}

    \pgfplotstableread{../test/CondEnc/OPT_HDdataL128_T.dat}{\table}
	\addplot[blue, mark = square*,  mark options={fill=white} ] table [x = {T}, y = {CondDec}] {\table};
	\label{OPTL128}

	\end{axis}
	
	\node [draw,fill=white] at (rel axis cs: 0.25,.77) {\shortstack[l]{
	       {\fontsize{30}{30}\selectfont $n = 8$ \ref{L8}, $n = 8$ \ref{OPTL8}}   \\
	       {\fontsize{30}{30}\selectfont $n = 16$ \ref{L16}, $n = 16$ \ref{OPTL16}} \\
		      {\fontsize{30}{30}\selectfont $n = 32$ \ref{L32}, $n = 32$ \ref{OPTL32}} \\
        {\fontsize{30}{30}\selectfont $n = 64$ \ref{L64}, $n = 64$ \ref{OPTL64}} \\
			{\fontsize{30}{30}\selectfont $n = 128$ \ref{L128}, $n = 128$ \ref{OPTL128}}\\
    {\fontsize{30}{30} \selectfont {\color{red} orig}, \fontsize{30}{30} \selectfont {\color{blue} optimized} }
		}};
	
	\end{tikzpicture}}}
\ffigbox[\FBwidth]{\caption{CEnc and Enc for HamDist}
	\figlab{fig:HDEnc}}
  {
  \resizebox{6cm}{!}{\begin{tikzpicture}
	\begin{axis}[tick label style={font=\fontsize{30}{30}\selectfont},
	xmin = 0, xmax = 140,
	ymin = 0, ymax = 30000,
	ymode=log,
	xtick distance = 16,
	ytick distance = 10,
	xlabel = {\fontsize{30}{30}\selectfont Input Length (n)},
	ylabel ={\fontsize{30}{30}\selectfont Time  (msec)},
	grid = both,
	minor tick num = 1,
	major grid style = {lightgray},
	minor grid style = {lightgray!25},
	width = 1.1\textwidth,
	height = .9\textwidth,
	%legend cell align = {left},
	legend style={at={(-2.3,-1.4)},anchor=north}
	%legend pos = north west
	]
	
	%\pgfplotstableread{ ../test/CondEnc/HDdataL8_T.dat}{\table}
	%\addplot[red, mark = *] table [x = {T}, y = {CondDec}] {\table};
	
	\pgfplotstableread{../test/CondEnc/HDdataL_T1.dat}{\table}
	\addplot[red, mark = square*] table [x = {L}, y = {Enc}] {\table};
	\label{p1}
	
	\pgfplotstableread{../test/CondEnc/HDdataL_T2.dat}{\table}
	\addplot[red, mark = triangle*] table [x = {L}, y = {Enc}] {\table};
	\label{p2}
	
	\pgfplotstableread{../test/CondEnc/HDdataL_T3.dat}{\table}
	\addplot[red, mark = *] table [x = {L}, y = {Enc}] {\table};
		\label{p3}
	\pgfplotstableread{../test/CondEnc/HDdataL_T4.dat}{\table}
	\addplot[red, mark = square*,  mark options={fill=white} ] table [x = {L}, y = {Enc}] {\table};
		\label{p4}
	
	\pgfplotstableread{../test/CondEnc/HDdataL_T1.dat}{\table}
	\addplot[blue, mark = square*] table [x = {L}, y = {CondEnc}] {\table};
	\label{p5}
	
	\pgfplotstableread{../test/CondEnc/HDdataL_T2.dat}{\table}
	\addplot[blue, mark = triangle*] table [x = {L}, y = {CondEnc}] {\table};
	\label{p6}
	
	\pgfplotstableread{../test/CondEnc/HDdataL_T3.dat}{\table}
	\addplot[blue, mark = *] table [x = {L}, y = {CondEnc}] {\table};
	\label{p7}
	
	\pgfplotstableread{../test/CondEnc/HDdataL_T4.dat}{\table}
	\addplot[blue, mark = square*,  mark options={fill=white} ] table [x = {L}, y = {CondEnc}] {\table};
	\label{p8}

	
	\end{axis}
	
	\node [draw,fill=white] at (rel axis cs: 0.75,0.1) {\shortstack[l]{
		 {\fontsize{30}{30}\selectfont $\ell = 1$: Enc \ref{p1}, CEnc \ref{p5}}  \\
		 {\fontsize{30}{30}\selectfont $\ell = 2$: Enc	\ref{p2}, CEnc  \ref{p6}}  \\
		 {\fontsize{30}{30}\selectfont $\ell = 3$: Enc \ref{p3}, CEnc \ref{p7}}   \\
		 {\fontsize{30}{30}\selectfont $\ell = 4$: Enc	\ref{p4}, CEnc \ref{p8}}  }};
	\end{tikzpicture}}}
\end{subfloatrow}

\begin{subfloatrow}[3]
\hspace{-0.15\textwidth}

\ffigbox[\FBwidth]{\caption{CEnc ED One}\figlab{EDOne}}{\resizebox{6cm}{!}{
\begin{tikzpicture}
	\begin{axis}[tick label style={font=\fontsize{30}{30}\selectfont},
	xmin = 0, xmax = 140,
	ymin = 0, ymax = 100000,
	ymode=log,
	xtick distance = 16,
	ytick distance = 10,
	xlabel = {\fontsize{30}{30}\selectfont Input Length (n)},
	ylabel ={\fontsize{30}{30}\selectfont Time  (msec)},
	grid = both,
	minor tick num = 1,
	major grid style = {lightgray},
	minor grid style = {lightgray!25},
	width = 1.1\textwidth,
	height = .9\textwidth,
	%legend cell align = {left},
	legend style={at={(0.1,-0.4)},anchor=north}
	%legend pos = north west
	]
	

	\pgfplotstableread{../test/CondEnc/EDOnedataL.dat}{\table}
	\addplot[red, mark = square*] table [x = {L}, y = {Enc}] {\table};
	\label{TradEnc}
	\pgfplotstableread{../test/CondEnc/EDOnedataL.dat}{\table}
	\addplot[black, mark = triangle*] table [x = {L}, y = {CondEnc}] {\table};
	\label{CEnc}

	\pgfplotstableread{../test/CondEnc/EDOnedataL.dat}{\table}
	\addplot[green, mark = *] table [x = {L}, y = {CondDec}] {\table};
	\label{CDec}

	\end{axis}
	\node [draw,fill=white] at (rel axis cs: .8,.05) {\shortstack[l]{
			\ref{TradEnc} {\fontsize{30}{30}\selectfont Enc }\\
			\ref{CEnc}  {\fontsize{30}{30}\selectfont CEnc} \\
			\ref{CDec} {\fontsize{30}{30}\selectfont CDec}}};
	\end{tikzpicture}}}

\ffigbox[\FBwidth]{	\caption{CEnc CAPSLCK ON}\label{CPALCK}\figlab{fig:CAPSLOCK}}{\resizebox{6cm}{!}{
\begin{tikzpicture}
	\begin{axis}[tick label style={font=\fontsize{30}{30}\selectfont},
	xmin = 0, xmax = 140,
	ymin = 0, ymax = 500,
	ymode=log,
    % xmode= log, 
	xtick distance = 16,
	ytick distance = 10,
	xlabel = {\fontsize{30}{30}\selectfont Input Length (n)},
	ylabel ={\fontsize{30}{30}\selectfont Time  (msec)},
	grid = both,
	minor tick num = 1,
	major grid style = {lightgray},
	minor grid style = {lightgray!25},
	width = 1.1\textwidth,
	height = .9\textwidth,
	%legend cell align = {left},
	legend style={at={(0.1,-0.4)},anchor=north}
	%legend pos = north west
	]
	
	
	\pgfplotstableread{../test/CondEnc/CAPSLKdataL.dat}{\table}
	\addplot[red, mark = square*] table [x = {L}, y = {Enc}] {\table};
	\label{TradEncCAPS}
	
	\pgfplotstableread{../test/CondEnc/CAPSLKdataL.dat}{\table}
	\addplot[black, mark = triangle*] table [x = {L}, y = {CondEnc}] {\table};
	\label{CEncCAPS}
	
	\pgfplotstableread{../test/CondEnc/CAPSLKdataL.dat}{\table}
	\addplot[green, mark = *] table [x = {L}, y = {CondDec}] {\table};
	\label{CDecCAPS}
	
	\end{axis}
	\node [draw,fill=white] at (rel axis cs: .8,.25) {\shortstack[l]{
			\ref{TradEncCAPS} {\fontsize{30}{30}\selectfont Enc} \\
			\ref{CEncCAPS}  {\fontsize{30}{30}\selectfont CEnc} \\
			\ref{CDecCAPS} {\fontsize{30}{30}\selectfont CDec}}};
	\end{tikzpicture}}}

\ffigbox[\FBwidth]{\caption{CEnc OR predicate} \figlab{fig:OR}\label{OR}}{\resizebox{6cm}{!}{	
\begin{tikzpicture}
	\begin{axis}[tick label style={font=\fontsize{30}{30}\selectfont},
	xmin = 0, xmax = 140,
	ymin = 0, ymax = 55000,
	ymode=log,
	xtick distance = 16,
	ytick distance = 10,
	xlabel = {\fontsize{30}{30}\selectfont Input Length (n)},
	ylabel ={\fontsize{30}{30}\selectfont Time  (msec)},
	grid = both,
	minor tick num = 1,
	major grid style = {lightgray},
	minor grid style = {lightgray!25},
	width = 1.1\textwidth,
	height = .9\textwidth,
	%legend cell align = {left},
	legend style={at={(0.1,-0.4)},anchor=north}
	%legend pos = north west
	]
	
	\pgfplotstableread{../test/CondEnc/ORdataL.dat}{\table}
	\addplot[red, mark = square*] table [x = {L}, y = {Enc}] {\table};
	\label{TradEncOR}
	
	\pgfplotstableread{../test/CondEnc/ORdataL.dat}{\table}
	\addplot[black, mark = triangle*] table [x = {L}, y = {CondEnc}] {\table};
	\label{CEncOR}
	
	\pgfplotstableread{../test/CondEnc/ORdataL.dat}{\table}
	\addplot[green, mark = *] table [x = {L}, y = {CondDec}] {\table};
	\label{CDecOR}
	
	
	\end{axis}
	\node [draw,fill=white] at (rel axis cs: 0.8,.0) {\shortstack[l]{
			\ref{TradEncOR} {\fontsize{30}{30}\selectfont Enc} \\
			\ref{CEncOR} {\fontsize{30}{30}\selectfont CEnc }\\
			\ref{CDecOR} {\fontsize{30}{30}\selectfont CDec}}};
		
	\end{tikzpicture}
}}

\end{subfloatrow}

\begin{subfloatrow}[3]
\hspace{-0.15\textwidth}

\ffigbox[\FBwidth]{\caption{CEnc OR predicate, CTX size} \figlab{fig:gCTXsiz}\label{OR_CtxtSize}}{\resizebox{6cm}{!}{	
\begin{tikzpicture}
	\begin{axis}[tick label style={font=\fontsize{30}{30}\selectfont},
	xmin = 0, xmax = 140,
	ymin = 0, ymax = 310000,
%	ymode=log,
	xtick distance = 16,
	ytick distance = 100000,
	xlabel = {\fontsize{30}{30}\selectfont Input Length (n)},
	ylabel = {\fontsize{30}{30}\selectfont Ciphertext Size (Bytes)},
	grid = both,
	minor tick num = 1,
	major grid style = {lightgray},
	minor grid style = {lightgray!25},
	width = 1.1\textwidth,
	height = .9\textwidth,
	%legend cell align = {left},
	legend style={at={(0.1,-0.4)},anchor=north}
	%legend pos = north west
	]
	
	\pgfplotstableread{../test/CondEnc/ORdataL.dat}{\table}
	\addplot[green, mark = square*] table [x = {L}, y = {CtxtSize}] {\table};
	\label{TradEncOR_size}
	
	\pgfplotstableread{../test/CondEnc/ORdataL.dat}{\table}
	\addplot[red, mark = triangle*] table [x = {L}, y = {CondCtxtSize}] {\table};
	\label{CEncOR_size}
	
	
	\end{axis}
	
	\node [draw,fill=white] at (rel axis cs: .8,.25) {\shortstack[l]{
			\ref{TradEncOR_size}{\fontsize{30}{30}\selectfont  Enc} \\
			\ref{CEncOR_size}  {\fontsize{30}{30}\selectfont CEnc} }}; 

	
	\end{tikzpicture}
}}


\ffigbox[\FBwidth]{\caption{CEnc HamDist, CTX size \figlab{fig:hCTXsize}} \label{HD_CTX Size}}{\resizebox{6cm}{!}{
\begin{tikzpicture}
	\begin{axis}[tick label style={font=\fontsize{30}{30}\selectfont},
	xmin = 0, xmax = 140,
	ymin = 0, ymax = 110000,
%	ymode=log,
	xtick distance = 16,
	ytick distance = 20000,
	xlabel = {\fontsize{30}{30}\selectfont Input Length (n)},
	ylabel = {\fontsize{30}{30}\selectfont Ciphertext Size (Bytes)},
	grid = both,
	minor tick num = 1,
	major grid style = {lightgray},
	minor grid style = {lightgray!25},
	width = 1.1\textwidth,
	height = .9\textwidth,
	%legend cell align = {left},
	legend style={at={(0.1,-0.4)},anchor=north}
	%legend pos = north west
	]
	
%	\pgfplotstableread{ ../test/CondEnc/HDdataL_T2.dat}{\table}
%	\addplot[blue, mark = triangle*] table [x = {L}, y = {CtxtSize}] {\table};
%	\label{TradHamDis_size}
	
	\pgfplotstableread{../test/CondEnc/HDdataL_T1.dat}{\table}
	\addplot[red,  mark = triangle*] table [x = {L}, y = {CtxtSize}] {\table};
	\label{HamDis_size_T1_Enc}
	
	\pgfplotstableread{../test/CondEnc/HDdataL_T1.dat}{\table}
	\addplot[blue,   mark = square*, mark options={fill=white}] table [x = {L}, y = {CondCtxtSize}] {\table};
	\label{HamDis_size_T1}
	
	\pgfplotstableread{../test/CondEnc/HDdataL_T2.dat}{\table}
	\addplot[blue,  mark = square*, mark options={fill=white}] table [x = {L}, y = {CondCtxtSize}] {\table};
	\label{HamDis_size_T2}
	\pgfplotstableread{../test/CondEnc/HDdataL_T2.dat}{\table}
	\addplot[red,mark = triangle* ] table [x = {L}, y = {CtxtSize}] {\table};
	\label{HamDis_size_T2_Enc}
	
	\pgfplotstableread{../test/CondEnc/HDdataL_T3.dat}{\table}
	\addplot[blue,  mark = square*, mark options={fill=white}] table [x = {L}, y = {CondCtxtSize}] {\table};
	\label{HamDis_size_T3}
	
	\pgfplotstableread{../test/CondEnc/HDdataL_T3.dat}{\table}
	\addplot[red, mark = triangle*] table [x = {L}, y = {CtxtSize}] {\table};
	\label{HamDis_size_T3_Enc}
	
	\pgfplotstableread{../test/CondEnc/HDdataL_T4.dat}{\table}
	\addplot[blue,mark = square*, mark options={fill=white}] table [x = {L}, y = {CondCtxtSize}] {\table};
	\label{HamDis_size_T4}
	
	\pgfplotstableread{../test/CondEnc/HDdataL_T4.dat}{\table}
	\addplot[red,  mark = triangle*] table [x = {L}, y = {CtxtSize}] {\table};
	\label{HamDis_size_T4_Enc}
	
	
	
	\end{axis}
	\node [draw,fill=white] at (rel axis cs: .45,.75) {\shortstack[l]{
			{\fontsize{30}{30}\selectfont $\ell=1$: CEnc \ref{HamDis_size_T1}, Enc \ref{HamDis_size_T1_Enc} }\\
			{\fontsize{30}{30}\selectfont $\ell=2$: CEnc \ref{HamDis_size_T2}, Enc \ref{HamDis_size_T2_Enc}} \\
			{\fontsize{30}{30}\selectfont $\ell=3$: CEnc \ref{HamDis_size_T3}, Enc \ref{HamDis_size_T3_Enc} }\\
			{\fontsize{30}{30}\selectfont $\ell=4$: CEnc \ref{HamDis_size_T4}, Enc \ref{HamDis_size_T4_Enc}	}}}; 
%			\ref{HamDis_size}  CEnc }}; 
	
	
	
	\end{tikzpicture}}}


\ffigbox[\FBwidth]{\caption{CEnc EDOne and CPSLCK, CTX size} \label{ED_CPSLCK_CTX Size} \figlab{fig:iCTXsize}}{\resizebox{6cm}{!}{
\begin{tikzpicture}
	\begin{axis}[tick label style={font=\fontsize{30}{30}\selectfont},
	xmin = 0, xmax = 140,
	ymin = 0, ymax = 360000,
	%ymode=log,
	xtick distance = 16,
	ytick distance = 100000,
	xlabel = {\fontsize{30}{30}\selectfont Input Length (n)},
	ylabel = {\fontsize{30}{30}\selectfont Ciphertext Size (Bytes)},
	grid = both,
	minor tick num = 1,
	major grid style = {lightgray},
	minor grid style = {lightgray!25},
	width = 1.1\textwidth,
	height = .9\textwidth,
	%legend cell align = {left},
	legend style={at={(0.1,-0.4)},anchor=north}
	%legend pos = north west
	]
	
	\pgfplotstableread{../test/CondEnc/AllCtxtSize.dat}{\table}
	\addplot[red,  mark = triangle*, mark options={fill=white}] table [x = {L}, y = {CtxtSizeOR}] {\table};
	\label{OR_Enc}
	
	\pgfplotstableread{../test/CondEnc/AllCtxtSize.dat}{\table}
	\addplot[blue,  mark = triangle*, mark options={fill=white}] table [x = {L}, y = {CondCtxtSizeOR}] {\table};
	\label{OR_CEnc}
	
	\pgfplotstableread{../test/CondEnc/AllCtxtSize.dat}{\table}
	\addplot[red,  mark = square*, mark options={fill=white}] table [x = {L}, y = {CtxtSizeHD}] {\table};
	\label{HD_Enc}
	
	\pgfplotstableread{../test/CondEnc/AllCtxtSize.dat}{\table}
	\addplot[blue,  mark = square*, mark options={fill=white}] table [x = {L}, y = {CondCtxtSizeHD}] {\table};
	\label{HD_CEnc}
	
	\pgfplotstableread{../test/CondEnc/AllCtxtSize.dat}{\table}
	\addplot[yellow,  mark = *,] table [x = {L}, y = {CtxtSizeED}] {\table};
	\label{ED_Enc}
	
	\pgfplotstableread{../test/CondEnc/AllCtxtSize.dat}{\table}
	\addplot[yellow,  mark = *, mark options={fill=white}] table [x = {L}, y = {CondCtxtSizeED}] {\table};
	\label{ED_CEnc}
	
	\pgfplotstableread{../test/CondEnc/AllCtxtSize.dat}{\table}
	\addplot[green,  mark = triangle*] table [x = {L}, y = {CtxtSizeCAPS}] {\table};
	\label{CAPS_Enc}
	
	\pgfplotstableread{../test/CondEnc/AllCtxtSize.dat}{\table}
	\addplot[orange,  mark =triangle*] table [x = {L}, y = {CondCtxtSizeCAPS}] {\table};
	\label{CAPS_CEnc}
	
	\end{axis}
		\node [draw,fill=white] at (rel axis cs: .3,.65) {\shortstack[l]{
			{\fontsize{30}{30}\selectfont OR: CEnc \ref{OR_CEnc}, Enc \ref{OR_Enc} }\\
			{\fontsize{30}{30}\selectfont HD: CEnc \ref{HD_CEnc}, Enc \ref{HD_Enc}} \\
			{\fontsize{30}{30}\selectfont ED: CEnc \ref{ED_CEnc}, Enc \ref{ED_CEnc}} \\
			{\fontsize{30}{30}\selectfont CAPS: CEnc \ref{CAPS_CEnc}, Enc \ref{CAPS_Enc}	}}}; 
	\end{tikzpicture}}}

\end{subfloatrow}} {\caption{Performance Evaluation for our proposed Conditional Encryption schemes under different, predicates, lengths and distances} \figlab{fig:FullEvalCondEnc}}
\end{center}
\end{figure*} 

% !TEX root =main.tex
%\begin {figure*}%[!hbtp]
%\begin{tikzpicture}
%\begin{axis}[
%xmin = 0, xmax = 5,
%ymin = 0, ymax = 10000000,
%ymode=log,
%xtick distance = 1,
%ytick distance = 1000,
%ylabel= Time (msec),
%xlabel=The maximum allowed hamming distance,
%grid = both,
%minor tick num = 1,
%major grid style = {lightgray},
%minor grid style = {lightgray!25},
%width = \linewidth,
%height = 0.75\textwidth,
%legend cell align = {left},
%legend pos = north west
%]
%
%%\pgfplotstableread{../Evaluation/CondEnc/HDdataL8_T.dat}{\table}
%%\addplot[red, mark = *] table [x = {T}, y = {CondDec}] {\table};
%
%\pgfplotstableread{../Evaluation/CondEnc/HDdataL16_T.dat}{\table}
%\addplot[red, mark =triangle*] table [x = {T}, y = {CondDec}] {\table};
%
%\pgfplotstableread{../Evaluation/CondEnc/HDdataL32_T.dat}{\table}
%\addplot[red, mark = square*] table [x = {T}, y = {CondDec}] {\table};
%
%\pgfplotstableread{../Evaluation/CondEnc/HDdataL64_T.dat}{\table}
%\addplot[red, mark = *] table [x = {T}, y = {CondDec}] {\table};
%
%\pgfplotstableread{../Evaluation/CondEnc/HDdataL128_T.dat}{\table}
%\addplot[red, mark =square*, mark options={fill=white}] table [x = {T}, y = {CondDec}] {\table};
%
%
%\pgfplotstableread{../Evaluation/CondEnc/HDdataL16_T.dat}{\table}
%\addplot[blue, mark =triangle*] table [x = {T}, y = {CondEnc}] {\table};
%
%\pgfplotstableread{../Evaluation/CondEnc/HDdataL32_T.dat}{\table}
%\addplot[blue, mark = square*] table [x = {T}, y = {CondEnc}] {\table};
%
%\pgfplotstableread{../Evaluation/CondEnc/HDdataL64_T.dat}{\table}
%\addplot[blue, mark = *] table [x = {T}, y = {CondEnc}] {\table};
%
%\pgfplotstableread{../Evaluation/CondEnc/HDdataL128_T.dat}{\table}
%\addplot[blue, mark  =square*, mark options={fill=white}] table [x = {T}, y = {CondEnc}] {\table};
%
%
%\pgfplotstableread{../Evaluation/CondEnc/HDdataL16_T.dat}{\table}
%\addplot[darkgreen, mark =triangle*] table [x = {T}, y = {Enc}] {\table};
%
%\pgfplotstableread{../Evaluation/CondEnc/HDdataL32_T.dat}{\table}
%\addplot[darkgreen, mark = square*] table [x = {T}, y = {Enc}] {\table};
%
%\pgfplotstableread{../Evaluation/CondEnc/HDdataL64_T.dat}{\table}
%\addplot[darkgreen, mark = *] table [x = {T}, y = {Enc}] {\table};
%
%\pgfplotstableread{../Evaluation/CondEnc/HDdataL128_T.dat}{\table}
%\addplot[darkgreen, mark  =square*, mark options={fill=white}] table [x = {T}, y = {Enc}] {\table};
%
%
%
%
%\legend{
%	Predicate: HD\_CondDec L=16,
%	Predicate: HD\_CondDec L=32, 
%	Predicate: HD\_CondDec L=64, 
%	Predicate: HD\_CondDec L=128, 
%	Predicate: HD\_CondEnc L=16,
%	Predicate: HD\_CondEnc L=32, 
%	Predicate: HD\_CondEnc L=64, 
%	Predicate: HD\_CondEnc L=128, 
%}
%\end{axis}
%\end{tikzpicture} 
%
%\end{figure*}
%
%\begin {figure*}%[!hbtp]
%\begin{tikzpicture}
%\begin{axis}[
%xmin = 0, xmax = 140,
%ymin = 0, ymax = 10000000,
%ymode=log,
%xtick distance = 10,
%ytick distance = 1000,
%xlabel = The input length,
%ylabel = Time (msec),
%grid = both,
%minor tick num = 1,
%major grid style = {lightgray},
%minor grid style = {lightgray!25},
%width = \linewidth,
%height = 0.75\textwidth,
%legend cell align = {left},
%legend pos = north west
%]
%
%%\pgfplotstableread{../Evaluation/CondEnc/HDdataL8_T.dat}{\table}
%%\addplot[red, mark = *] table [x = {T}, y = {CondDec}] {\table};
%
%\pgfplotstableread{../Evaluation/CondEnc/HDdataL_T1.dat}{\table}
%\addplot[red, mark = square*] table [x = {L}, y = {CondDec}] {\table};
%
%\pgfplotstableread{../Evaluation/CondEnc/HDdataL_T2.dat}{\table}
%\addplot[red, mark = triangle*] table [x = {L}, y = {CondDec}] {\table};
%
%\pgfplotstableread{../Evaluation/CondEnc/HDdataL_T3.dat}{\table}
%\addplot[red, mark = *] table [x = {L}, y = {CondDec}] {\table};
%
%\pgfplotstableread{../Evaluation/CondEnc/HDdataL_T4.dat}{\table}
%\addplot[red, mark = square*,  mark options={fill=white} ] table [x = {L}, y = {CondDec}] {\table};
%
%\pgfplotstableread{../Evaluation/CondEnc/HDdataL_T1.dat}{\table}
%\addplot[blue, mark = square*] table [x = {L}, y = {CondEnc}] {\table};
%
%\pgfplotstableread{../Evaluation/CondEnc/HDdataL_T2.dat}{\table}
%\addplot[blue, mark = triangle*] table [x = {L}, y = {CondEnc}] {\table};
%
%\pgfplotstableread{../Evaluation/CondEnc/HDdataL_T3.dat}{\table}
%\addplot[blue, mark = *] table [x = {L}, y = {CondEnc}] {\table};
%
%\pgfplotstableread{../Evaluation/CondEnc/HDdataL_T4.dat}{\table}
%\addplot[blue, mark = square*,  mark options={fill=white} ] table [x = {L}, y = {CondEnc}] {\table};
%
%
%\legend{
%	Predicate: HD\_CondDec T=1,
%	Predicate: HD\_CondDec T=2, 
%	Predicate: HD\_CondDec T=3, 
%	Predicate: HD\_CondDec T=4, 
%	Predicate: HD\_CondEnc T=1,
%	Predicate: HD\_CondEnc T=2, 
%	Predicate: HD\_CondEnc T=3, 
%	Predicate: HD\_CondEnc T=4, 
%}
%\end{axis}
%\end{tikzpicture} 
%
%\end{figure*}
\begin{table*}
\begin{figure}

\captionsetup{justification=raggedright}
\ffigbox[\textwidth]{%
\begin{subfloatrow}[3]
\hspace{-0.3\textwidth}\ffigbox[\FBwidth]{\caption{CDec Time for HamDist}
		\label{CondDecHD}
		\figlab{fig:HDdifT}}
  {
  \resizebox{6cm}{!}{\begin{tikzpicture}
		\begin{axis}[
		xmin = 0, xmax = 140,
		ymin = 0, ymax = 100000000000,
		ymode=log,
		xtick distance = 32,
		ytick distance = 1000,
		xlabel = The input length,
		ylabel =Time  (msec),
		grid = both,
		minor tick num = 1,
		major grid style = {lightgray},
		minor grid style = {lightgray!25},
		width = 1.1\textwidth,
		height = .9\textwidth,
		%legend cell align = {left},
		legend style={at={(0.1,-0.4)},anchor=north}
		%legend pos = north west
		]
		
		%\pgfplotstableread{../Evaluation/CondEnc/HDdataL8_T.dat}{\table}
		%\addplot[red, mark = *] table [x = {T}, y = {CondDec}] {\table};
		
		\pgfplotstableread{Evaluation/CondEnc/HDdataL_T1.dat}{\table}
		\addplot[red, mark = square*] table [x = {L}, y = {CondDec}] {\table};
		\label{T1}
  
        \pgfplotstableread{Evaluation/CondEnc/OPT_HDdataL_T1.dat}{\table}
		\addplot[blue, mark = square*] table [x = {L}, y = {CondDec}] {\table};
		\label{OPTT1}
		
		\pgfplotstableread{Evaluation/CondEnc/HDdataL_T2.dat}{\table}
		\addplot[red, mark = triangle*] table [x = {L}, y = {CondDec}] {\table};
		\label{T2}

        \pgfplotstableread{Evaluation/CondEnc/OPT_HDdataL_T2.dat}{\table}
		\addplot[blue, mark = triangle*] table [x = {L}, y = {CondDec}] {\table};
		\label{OPTT2}
		
		\pgfplotstableread{Evaluation/CondEnc/HDdataL_T3.dat}{\table}
		\addplot[red, mark = *] table [x = {L}, y = {CondDec}] {\table};
		\label{T3}

        \pgfplotstableread{Evaluation/CondEnc/OPT_HDdataL_T3.dat}{\table}
		\addplot[blue, mark = *] table [x = {L}, y = {CondDec}] {\table};
		\label{OPTT3}
		
		\pgfplotstableread{Evaluation/CondEnc/HDdataL_T4.dat}{\table}
		\addplot[red, mark = square*,  mark options={fill=white} ] table [x = {L}, y = {CondDec}] {\table};
		\label{T4}

        \pgfplotstableread{Evaluation/CondEnc/OPT_HDdataL_T4.dat}{\table}
		\addplot[blue, mark = square*,  mark options={fill=white} ] table [x = {L}, y = {CondDec}] {\table};
		\label{OPTT4}

		\end{axis}
		
			\node [draw,fill=white] at (rel axis cs: 0.8,.8) {\shortstack[l]{
				T = 1 \ref{T1}, T = 1 \ref{OPTT1}   \\
				T = 2 \ref{T2}, T = 2 \ref{OPTT2}  \\
				T = 3 \ref{T3}, T = 3 \ref{OPTT3}  \\
				T = 4 \ref{T4}, T = 4 \ref{OPTT4} }};
		
		\end{tikzpicture}}}
\ffigbox[\FBwidth]{\caption{CDec Time for HamDist}\figlab{fig:HDdifL}\label{CondDecHD_T}}
  {
  \resizebox{6cm}{!}{\begin{tikzpicture}
	\begin{axis}[
	xmin = 0, xmax = 5,
	ymin = 0, ymax = 10000000000,
	ymode=log,
	xtick distance = 1,
	ytick distance = 1000,
	xlabel = Max Haming Distance,
	ylabel =Time  (msec),
	grid = both,
	minor tick num = 1,
	major grid style = {lightgray},
	minor grid style = {lightgray!25},
	width = 1.1\textwidth,
	height = .9\textwidth,
	%legend cell align = {left},
	legend style={at={(0.1,-0.25)},anchor=north}
	%legend pos = north west
	]
	
	\pgfplotstableread{Evaluation/CondEnc/HDdataL8_T.dat}{\table}
	\addplot[red, mark = *] table [x = {T}, y = {CondDec}] {\table};
	\label{L8}

    \pgfplotstableread{Evaluation/CondEnc/OPT_HDdataL8_T.dat}{\table}
	\addplot[blue, mark = *] table [x = {T}, y = {CondDec}] {\table};
	\label{OPTL8}
	
	\pgfplotstableread{Evaluation/CondEnc/HDdataL16_T.dat}{\table}
	\addplot[red, mark = square*] table [x = {T}, y = {CondDec}] {\table};
	\label{L16}
 
 \pgfplotstableread{Evaluation/CondEnc/OPT_HDdataL16_T.dat}{\table}
	\addplot[blue, mark = square*] table [x = {T}, y = {CondDec}] {\table};
	\label{OPTL16}
	
	\pgfplotstableread{Evaluation/CondEnc/HDdataL32_T.dat}{\table}
	\addplot[red, mark = triangle*] table [x = {T}, y = {CondDec}] {\table};
	\label{L32}

    \pgfplotstableread{Evaluation/CondEnc/OPT_HDdataL32_T.dat}{\table}
	\addplot[blue, mark = triangle*] table [x = {T}, y = {CondDec}] {\table};
	\label{OPTL32}
 
	
	\pgfplotstableread{Evaluation/CondEnc/HDdataL64_T.dat}{\table}
	\addplot[red, mark = *] table [x = {T}, y = {CondDec}] {\table};
	\label{L64}

    \pgfplotstableread{Evaluation/CondEnc/OPT_HDdataL64_T.dat}{\table}
	\addplot[blue, mark = *] table [x = {T}, y = {CondDec}] {\table};
	\label{OPTL64}
	
	\pgfplotstableread{Evaluation/CondEnc/HDdataL128_T.dat}{\table}
	\addplot[red, mark = square*,  mark options={fill=white} ] table [x = {T}, y = {CondDec}] {\table};
	\label{L128}

    \pgfplotstableread{Evaluation/CondEnc/OPT_HDdataL128_T.dat}{\table}
	\addplot[blue, mark = square*,  mark options={fill=white} ] table [x = {T}, y = {CondDec}] {\table};
	\label{OPTL128}

	\end{axis}
	
	\node [draw,fill=white] at (rel axis cs: 0.25,.8) {\shortstack[l]{
	       L = 8 \ref{L8}, L = 8 \ref{OPTL8}   \\
	       L = 16 \ref{L16}, L = 16 \ref{OPTL16} \\
		      L = 32 \ref{L32}, L = 32 \ref{OPTL32} \\
        L = 64 \ref{L64}, L = 64 \ref{OPTL64}\\
			L = 128 \ref{L128}, L = 128 \ref{OPTL128}
		}};
	
	\end{tikzpicture}}}
\ffigbox[\FBwidth]{\caption{CEnc and Enc for HamDist}
	\figlab{fig:HDCtxtSize}}
  {
  \resizebox{6cm}{!}{\begin{tikzpicture}
	\begin{axis}[
	xmin = 0, xmax = 140,
	ymin = 0, ymax = 30000,
	ymode=log,
	xtick distance = 16,
	ytick distance = 10,
	xlabel = The input length,
	ylabel =Time  (msec),
	grid = both,
	minor tick num = 1,
	major grid style = {lightgray},
	minor grid style = {lightgray!25},
	width = 1.1\textwidth,
	height = .9\textwidth,
	%legend cell align = {left},
	legend style={at={(-2.3,-1.4)},anchor=north}
	%legend pos = north west
	]
	
	%\pgfplotstableread{ Evaluation/CondEnc/HDdataL8_T.dat}{\table}
	%\addplot[red, mark = *] table [x = {T}, y = {CondDec}] {\table};
	
	\pgfplotstableread{Evaluation/CondEnc/HDdataL_T1.dat}{\table}
	\addplot[red, mark = square*] table [x = {L}, y = {Enc}] {\table};
	\label{p1}
	
	\pgfplotstableread{Evaluation/CondEnc/HDdataL_T2.dat}{\table}
	\addplot[red, mark = triangle*] table [x = {L}, y = {Enc}] {\table};
	\label{p2}
	
	\pgfplotstableread{Evaluation/CondEnc/HDdataL_T3.dat}{\table}
	\addplot[red, mark = *] table [x = {L}, y = {Enc}] {\table};
		\label{p3}
	\pgfplotstableread{Evaluation/CondEnc/HDdataL_T4.dat}{\table}
	\addplot[red, mark = square*,  mark options={fill=white} ] table [x = {L}, y = {Enc}] {\table};
		\label{p4}
	
	\pgfplotstableread{Evaluation/CondEnc/HDdataL_T1.dat}{\table}
	\addplot[blue, mark = square*] table [x = {L}, y = {CondEnc}] {\table};
	\label{p5}
	
	\pgfplotstableread{Evaluation/CondEnc/HDdataL_T2.dat}{\table}
	\addplot[blue, mark = triangle*] table [x = {L}, y = {CondEnc}] {\table};
	\label{p6}
	
	\pgfplotstableread{Evaluation/CondEnc/HDdataL_T3.dat}{\table}
	\addplot[blue, mark = *] table [x = {L}, y = {CondEnc}] {\table};
	\label{p7}
	
	\pgfplotstableread{Evaluation/CondEnc/HDdataL_T4.dat}{\table}
	\addplot[blue, mark = square*,  mark options={fill=white} ] table [x = {L}, y = {CondEnc}] {\table};
	\label{p8}

	
	\end{axis}
	
	\node [draw,fill=white] at (rel axis cs: 0.35,0.8) {\shortstack[l]{
		 T = 1: Enc \ref{p1}, CEnc \ref{p5}  \\
		 T = 2: Enc	\ref{p2}, CEnc  \ref{p6}  \\
		 T = 3: Enc \ref{p3}, CEnc \ref{p7}   \\
		 T = 4: Enc	\ref{p4}, CEnc \ref{p8}  }};

	\end{tikzpicture}}}
\end{subfloatrow}

\begin{subfloatrow}[3]
\hspace{-0.3\textwidth}

\ffigbox[\FBwidth]{\caption{CEnc ED One}\figlab{EDOne}}{\resizebox{6cm}{!}{
\begin{tikzpicture}
	\begin{axis}[
	xmin = 0, xmax = 140,
	ymin = 0, ymax = 100000,
	ymode=log,
	xtick distance = 16,
	ytick distance = 10,
	xlabel = The input length,
	ylabel =Time  (msec),
	grid = both,
	minor tick num = 1,
	major grid style = {lightgray},
	minor grid style = {lightgray!25},
	width = 1.1\textwidth,
	height = .9\textwidth,
	%legend cell align = {left},
	legend style={at={(0.1,-0.4)},anchor=north}
	%legend pos = north west
	]
	

	\pgfplotstableread{Evaluation/CondEnc/EDOnedataL.dat}{\table}
	\addplot[red, mark = square*] table [x = {L}, y = {Enc}] {\table};
	\label{TradEnc}
	\pgfplotstableread{Evaluation/CondEnc/EDOnedataL.dat}{\table}
	\addplot[red, mark = triangle*] table [x = {L}, y = {CondEnc}] {\table};
	\label{CEnc}

	\pgfplotstableread{Evaluation/CondEnc/EDOnedataL.dat}{\table}
	\addplot[red, mark = *] table [x = {L}, y = {CondDec}] {\table};
	\label{CDec}

	\end{axis}
	\node [draw,fill=white] at (rel axis cs: .8,.05) {\shortstack[l]{
			\ref{TradEnc} Enc \\
			\ref{CEnc}  CEnc \\
			\ref{CDec} CDec}};
	\end{tikzpicture}}}

\ffigbox[\FBwidth]{	\caption{CEnc CAPSLCK ON}\label{CPALCK}\figlab{fig:CAPSLOCK}}{\resizebox{6cm}{!}{
\begin{tikzpicture}
	\begin{axis}[
	xmin = 0, xmax = 140,
	ymin = 0, ymax = 500,
	ymode=log,
    % xmode= log, 
	xtick distance = 16,
	ytick distance = 10,
	xlabel = The input length,
	ylabel =Time  (msec),
	grid = both,
	minor tick num = 1,
	major grid style = {lightgray},
	minor grid style = {lightgray!25},
	width = 1.1\textwidth,
	height = .9\textwidth,
	%legend cell align = {left},
	legend style={at={(0.1,-0.4)},anchor=north}
	%legend pos = north west
	]
	
	
	\pgfplotstableread{Evaluation/CondEnc/CAPSLKdataL.dat}{\table}
	\addplot[red, mark = square*] table [x = {L}, y = {Enc}] {\table};
	\label{TradEncCAPS}
	
	\pgfplotstableread{Evaluation/CondEnc/CAPSLKdataL.dat}{\table}
	\addplot[red, mark = triangle*] table [x = {L}, y = {CondEnc}] {\table};
	\label{CEncCAPS}
	
	\pgfplotstableread{Evaluation/CondEnc/CAPSLKdataL.dat}{\table}
	\addplot[red, mark = *] table [x = {L}, y = {CondDec}] {\table};
	\label{CDecCAPS}
	
	\end{axis}
	\node [draw,fill=white] at (rel axis cs: .8,.25) {\shortstack[l]{
			\ref{TradEncCAPS} Enc \\
			\ref{CEncCAPS}  CEnc \\
			\ref{CDecCAPS} CDec}};
	\end{tikzpicture}}}

\ffigbox[\FBwidth]{\caption{CEnc OR predicate}\label{OR}}{\resizebox{6cm}{!}{	
\begin{tikzpicture}
	\begin{axis}[
	xmin = 0, xmax = 140,
	ymin = 0, ymax = 55000,
	ymode=log,
	xtick distance = 16,
	ytick distance = 10,
	xlabel = The input length,
	ylabel =Time  (msec),
	grid = both,
	minor tick num = 1,
	major grid style = {lightgray},
	minor grid style = {lightgray!25},
	width = 1.1\textwidth,
	height = .9\textwidth,
	%legend cell align = {left},
	legend style={at={(0.1,-0.4)},anchor=north}
	%legend pos = north west
	]
	
	\pgfplotstableread{Evaluation/CondEnc/ORdataL.dat}{\table}
	\addplot[red, mark = square*] table [x = {L}, y = {Enc}] {\table};
	\label{TradEncOR}
	
	\pgfplotstableread{Evaluation/CondEnc/ORdataL.dat}{\table}
	\addplot[red, mark = triangle*] table [x = {L}, y = {CondEnc}] {\table};
	\label{CEncOR}
	
	\pgfplotstableread{Evaluation/CondEnc/ORdataL.dat}{\table}
	\addplot[red, mark = *] table [x = {L}, y = {CondDec}] {\table};
	\label{CDecOR}
	
	
	\end{axis}
	\node [draw,fill=white] at (rel axis cs: 0.8,.0) {\shortstack[l]{
			\ref{TradEncOR} Enc \\
			\ref{CEncOR}  CEnc \\
			\ref{CDecOR} CDec}};
		
	\end{tikzpicture}
}}

\end{subfloatrow}

\begin{subfloatrow}[3]
\hspace{-0.3\textwidth}

\ffigbox[\FBwidth]{\caption{CEnc OR predicate, CTX size} \label{OR_CtxtSize}}{\resizebox{6cm}{!}{	
\begin{tikzpicture}
	\begin{axis}[
	xmin = 0, xmax = 140,
	ymin = 0, ymax = 310000,
%	ymode=log,
	xtick distance = 16,
	ytick distance = 100000,
	xlabel = The input length,
	ylabel = Ciphertext Size (Bytes),
	grid = both,
	minor tick num = 1,
	major grid style = {lightgray},
	minor grid style = {lightgray!25},
	width = 1.1\textwidth,
	height = .9\textwidth,
	%legend cell align = {left},
	legend style={at={(0.1,-0.4)},anchor=north}
	%legend pos = north west
	]
	
	\pgfplotstableread{Evaluation/CondEnc/ORdataL.dat}{\table}
	\addplot[red, mark = square*] table [x = {L}, y = {CtxtSize}] {\table};
	\label{TradEncOR_size}
	
	\pgfplotstableread{Evaluation/CondEnc/ORdataL.dat}{\table}
	\addplot[red, mark = triangle*] table [x = {L}, y = {CondCtxtSize}] {\table};
	\label{CEncOR_size}
	
	
	\end{axis}
	
	\node [draw,fill=white] at (rel axis cs: .8,.25) {\shortstack[l]{
			\ref{TradEncOR_size} Enc \\
			\ref{CEncOR_size}  CEnc }}; 

	
	\end{tikzpicture}
}}


\ffigbox[\FBwidth]{\caption{CEnc HamDist, CTX size} \label{HD_CTX Size}}{\resizebox{6cm}{!}{
\begin{tikzpicture}
	\begin{axis}[
	xmin = 0, xmax = 140,
	ymin = 0, ymax = 110000,
%	ymode=log,
	xtick distance = 16,
	ytick distance = 20000,
	xlabel = The input length,
	ylabel = Ciphertext Size (Bytes),
	grid = both,
	minor tick num = 1,
	major grid style = {lightgray},
	minor grid style = {lightgray!25},
	width = 1.1\textwidth,
	height = .9\textwidth,
	%legend cell align = {left},
	legend style={at={(0.1,-0.4)},anchor=north}
	%legend pos = north west
	]
	
%	\pgfplotstableread{ Evaluation/CondEnc/HDdataL_T2.dat}{\table}
%	\addplot[blue, mark = triangle*] table [x = {L}, y = {CtxtSize}] {\table};
%	\label{TradHamDis_size}
	
	\pgfplotstableread{Evaluation/CondEnc/HDdataL_T1.dat}{\table}
	\addplot[red,  mark = triangle*] table [x = {L}, y = {CtxtSize}] {\table};
	\label{HamDis_size_T1_Enc}
	
	\pgfplotstableread{Evaluation/CondEnc/HDdataL_T1.dat}{\table}
	\addplot[blue,   mark = square*, mark options={fill=white}] table [x = {L}, y = {CondCtxtSize}] {\table};
	\label{HamDis_size_T1}
	
	\pgfplotstableread{Evaluation/CondEnc/HDdataL_T2.dat}{\table}
	\addplot[blue,  mark = square*, mark options={fill=white}] table [x = {L}, y = {CondCtxtSize}] {\table};
	\label{HamDis_size_T2}
	\pgfplotstableread{Evaluation/CondEnc/HDdataL_T2.dat}{\table}
	\addplot[red,mark = triangle* ] table [x = {L}, y = {CtxtSize}] {\table};
	\label{HamDis_size_T2_Enc}
	
	\pgfplotstableread{Evaluation/CondEnc/HDdataL_T3.dat}{\table}
	\addplot[blue,  mark = square*, mark options={fill=white}] table [x = {L}, y = {CondCtxtSize}] {\table};
	\label{HamDis_size_T3}
	
	\pgfplotstableread{Evaluation/CondEnc/HDdataL_T3.dat}{\table}
	\addplot[red, mark = triangle*] table [x = {L}, y = {CtxtSize}] {\table};
	\label{HamDis_size_T3_Enc}
	
	\pgfplotstableread{Evaluation/CondEnc/HDdataL_T4.dat}{\table}
	\addplot[blue,mark = square*, mark options={fill=white}] table [x = {L}, y = {CondCtxtSize}] {\table};
	\label{HamDis_size_T4}
	
	\pgfplotstableread{Evaluation/CondEnc/HDdataL_T4.dat}{\table}
	\addplot[red,  mark = triangle*] table [x = {L}, y = {CtxtSize}] {\table};
	\label{HamDis_size_T4_Enc}
	
	
	
	\end{axis}
	\node [draw,fill=white] at (rel axis cs: .45,.75) {\shortstack[l]{
			T=1: CEnc \ref{HamDis_size_T1}, Enc \ref{HamDis_size_T1_Enc} \\
			T=2: CEnc \ref{HamDis_size_T2}, Enc \ref{HamDis_size_T2_Enc} \\
			T=3: CEnc \ref{HamDis_size_T3}, Enc \ref{HamDis_size_T3_Enc} \\
			T=4: CEnc \ref{HamDis_size_T4}, Enc \ref{HamDis_size_T4_Enc}	}}; 
%			\ref{HamDis_size}  CEnc }}; 
	
	
	
	\end{tikzpicture}}}


\ffigbox[\FBwidth]{\caption{CEnc EDOne and CPSLCK, CTX size} \label{ED_CPSLCK_CTX Size}}{\resizebox{6cm}{!}{
\begin{tikzpicture}
	\begin{axis}[
	xmin = 0, xmax = 140,
	ymin = 0, ymax = 360000,
	%ymode=log,
	xtick distance = 16,
	ytick distance = 100000,
	xlabel = The input length,
	ylabel = Ciphertext Size (Bytes),
	grid = both,
	minor tick num = 1,
	major grid style = {lightgray},
	minor grid style = {lightgray!25},
	width = 1.1\textwidth,
	height = .9\textwidth,
	%legend cell align = {left},
	legend style={at={(0.1,-0.4)},anchor=north}
	%legend pos = north west
	]
	
	\pgfplotstableread{Evaluation/CondEnc/AllCtxtSize.dat}{\table}
	\addplot[red,  mark = triangle*, mark options={fill=white}] table [x = {L}, y = {CtxtSizeOR}] {\table};
	\label{OR_Enc}
	
	\pgfplotstableread{Evaluation/CondEnc/AllCtxtSize.dat}{\table}
	\addplot[blue,  mark = triangle*, mark options={fill=white}] table [x = {L}, y = {CondCtxtSizeOR}] {\table};
	\label{OR_CEnc}
	
	\pgfplotstableread{Evaluation/CondEnc/AllCtxtSize.dat}{\table}
	\addplot[red,  mark = square*, mark options={fill=white}] table [x = {L}, y = {CtxtSizeHD}] {\table};
	\label{HD_Enc}
	
	\pgfplotstableread{Evaluation/CondEnc/AllCtxtSize.dat}{\table}
	\addplot[blue,  mark = square*, mark options={fill=white}] table [x = {L}, y = {CondCtxtSizeHD}] {\table};
	\label{HD_CEnc}
	
	\pgfplotstableread{Evaluation/CondEnc/AllCtxtSize.dat}{\table}
	\addplot[yellow,  mark = *,] table [x = {L}, y = {CtxtSizeED}] {\table};
	\label{ED_Enc}
	
	\pgfplotstableread{Evaluation/CondEnc/AllCtxtSize.dat}{\table}
	\addplot[yellow,  mark = *, mark options={fill=white}] table [x = {L}, y = {CondCtxtSizeED}] {\table};
	\label{ED_CEnc}
	
	\pgfplotstableread{Evaluation/CondEnc/AllCtxtSize.dat}{\table}
	\addplot[green,  mark = triangle*] table [x = {L}, y = {CtxtSizeCAPS}] {\table};
	\label{CAPS_Enc}
	
	\pgfplotstableread{Evaluation/CondEnc/AllCtxtSize.dat}{\table}
	\addplot[orange,  mark =triangle*] table [x = {L}, y = {CondCtxtSizeCAPS}] {\table};
	\label{CAPS_CEnc}
	
	\end{axis}
		\node [draw,fill=white] at (rel axis cs: .2,.65) {\shortstack[l]{
			OR: CEnc \ref{OR_CEnc}, Enc \ref{OR_Enc} \\
			HD: CEnc \ref{HD_CEnc}, Enc \ref{HD_Enc} \\
			ED: CEnc \ref{ED_CEnc}, Enc \ref{ED_CEnc} \\
			CAPS: CEnc \ref{CAPS_CEnc}, Enc \ref{CAPS_Enc}	}}; 
	\end{tikzpicture}}}

\end{subfloatrow}} {\caption{Performance Evaluation for our proposed Conditional Encryption schemes under different, predicates, lengths and distances} \figlab{fig:FullEvalCondEnc}}

\end{figure}
\end{table*}
 




\bibliographystyle{ACM-Reference-Format}
\bibliography{extra,cryptobib/abbrev0,cryptobib/crypto}


%%
%% If your work has an appendix, this is the place to put it.
\appendix

\input{prelimsApndx.tex} 
\input{RealOrRandomDefProosAppendix}
\input{FHEDiscussion}
\section{Missing Proofs}
% \label{apdx:MissingProofs}
\applab{apdx:MissingProofs}

\begin{remindertheorem} {\thmref{thm:EqCor}}
\ThmEqCorrectness
\end{remindertheorem}
\begin{proofof}
{\thmref{thm:EqCor}}
 Since $\Enc_{pk}(m)$ simply runs regular Pallier encryption perfect correctness of Pallier immediately implies that 
 $\Dec_{sk}\left( \Enc_{pk}(m)\right) =  \Dec_{sk}\left(P.\Enc_{pk}(\ToInt(m) \right)= \ToOrig$ $(\ToInt(m)) = m$ with probability $1$ for all messages $m \in \Sigma^{\leq n}$ and all public/private key pairs in the support of $\KG$. Similarly, if $c_1=\Enc_{pk}(m)$ and $P_{=}(m_1,m_2) = 1$ then $\Cond\Enc_{pk}(c_1,m_2,m_3)$ will output $(1,c')$ where $c' = g^{\ToInt{(m_3)}} r^n \mod{N^2}$ for some $r \in \mathbb{Z}_N^*$. Thus, $c'$ is a valid pallier ciphertext for $\ToInt{(m_3)}$ and, by correctness of Pallier, $\Dec_{sk}(1,c')$ will return $m_3$. 

 On the other hand if $P_{=}(m_1,m_2) = 0$ then by \thmref{thm:EqualityTestSecrecy} the ciphertext $c'$ is a valid Pallier Ciphertext for some uniformly random integer $y \in \mathbb{Z}_N$ and we will have $\Dec_{sk}(1,c') = \bot$ as long as $y > |\Sigma|^{n+1}$. Thus, the construction is $1-\epsilon(\lambda)$-error detecting conditional encryption scheme with $\eps(\lambda) = \frac{|\Sigma|^{n+1}}{N} \leq \frac{1}{\max\{p,q\}}$.
\end{proofof}



\begin{remindertheorem}{\thmref{thm:EqualityTestSecrecy}}  
\ThmEqTestSecrecy
\end{remindertheorem}
\begin{proofof}{\thmref{thm:EqualityTestSecrecy}}
We define the simulator $ \Sim(pk) $ as follows. The simulator $ \Sim(pk) $ takes as input the Paillier public key $ pk $ and then selects $ R_s\in_R \mathbb{Z}_{N} $ and $r_s\in_R \mathbb{Z}^*_N$ uniformly at random and then encrypts $ R_s $ as $ C_{\Sim} = \Pail.\Enc_{pk}(R_s; r_s) = g^{R_s} r_s^{N} \mod N^2$ and outputs it. We now argue that for any $m_1, m_2 \in \Sigma^{n}$ with $P_{=}(m_1,m_2) = 0$ any payload message $m_3$ and any Pallier key $\left(pk=\left(N=pq,g\right),sk\right)$  which satisfies our condition that $|\Sigma|^{n+1} \leq \min\{p,q\}$ and any encryption $c_1 = g^{m_1} r_1^{N} \mod{N^2}$ of $m_1$ under $pk$ that the distributions $ \left(pk, sk, m_1, m_2, m_3, c_1, C_{m_3} = \Cond\Enc_{pk}\left(c_{m_1}, m_2, m_3\right) \right)$ and $ (pk, sk,$ $ m_1, m_2, m_3,c_1, C_{\Sim}= \Sim(pk)) $ are identical. In particular, it suffices to argue that $C_{\Sim}= \Sim(pk)$ and $\Cond\Enc_{pk}(c_{m_1}, m_2, m_3)$ are distributed identically. 

To see this consider the generation of $\Cond\Enc_{pk}(c_{m_1}, m_2, m_3)$. First, we pick a random $R \in \mathbb{Z}_N$ and generate an encryption of $(-R \cdot m_2 \mod N)$ as $c_2 = g^{-R \cdot m_2} r_2^N \mod{N^2}$ where $r_2 \in \mathbb{Z}_N^*$ is picked randomly. We then compute $c_1^R = g^{m_1 \cdot R} r_1^{RN}$. Finally, we output
\begin{eqnarray*}&&c_1^Rc_2 \cdot g^{m_3} \mod{N^2} =  g^{m_3+R(m_1-m_2)} r_1^{RN} r_2^N \\ 
&&= g^{m_3+R(m_1-m_2) \mod N} \left(r_1^Rr_2 \mod{N}\right)^{N} \mod{N^2} \ . \end{eqnarray*}
where the values $R \in \mathbb{Z}_N$, $r_2 \in \mathbb{Z}_N^*$ are fresh random values. In the last step we implicitly used the fact that if $r_1^Rr_2 = aN+b$ where $b = \left[r_1^Rr_2 \mod{N}\right]$ then \[ (aN+b)^N = \sum_{i=0}^N {N \choose i} (aN)^ib^{N-i} = b^{N} \mod{N^2} \ . \]

Let us first focus on the term $m_3+R(m_1-m_2)$ in the exponent of $g$. We observe that $[m_1-m_2 \mod N] \in \mathbb{Z}_N^*$ since $1 \leq |m_1-m_2| \leq \left| \Sigma\right|^{n} \leq \min\{p,q\}$. It follows that for any $m_3$ that $R(m_1-m_2) + m_3$ is distributed uniformly at random in $\mathbb{Z}_N$ when $R \in \mathbb{Z}_N$ is picked randomly. We next consider the term $\left(r_1^Rr_2 \mod{N}\right)$ and argue for any fixed $r_1 \in \mathbb{Z}_N^*$ and $R \in \mathbb{Z}_N$ that $\left(r_1^Rr_2 \mod{N}\right)$ is distributed uniformly at random in $\mathbb{Z}_N^*$ when $r_2 \in \mathbb{Z}_N^*$ is picked randomly. It follows that for any $r_1 \in \mathbb{Z}_N^*, m_1,m_2,m_3$ such that $1 \leq |m_1-m_2| \leq \min\{p,q\}$ that the simulated ciphertext ($C_{\Sim}= \Sim(pk) = g^{R_s} r_s^N \mod{N^2}$ for random $r_s \in \mathbb{Z}_N^*$ and $R_s \in \mathbb{Z}_N$ ) is identically distributed to $\Cond\Enc_{pk}(c_{m_1}, m_2, m_3)$.
\end{proofof}



\begin{remindertheorem}{\thmref{thm:StatDistUak}}  
\thmStatDistTwo
\end{remindertheorem}
\begin{proofof} {\thmref{thm:StatDistUak}}
    ‌Based on the definition of statistical distance we have 
    \begin{align}
        \mathtt{SD}(\D_{ak}, \U_b) &= \frac{1}{2} \sum_{i = 0}^{b-1} |\Pr_{y\in_R \D_{ak}}[y = i] - \Pr_{y\in_R \U_b}[y = i]| \nonumber\\ 
        & = \frac{1}{2} (ak) (| \frac{1}{b} - \frac{1}{a} \cdot\frac{1}{k}| + (b-ka) (|\frac{1}{b} - 0|)\nonumber\\ 
        & = \frac{1}{2}(ak) (\frac{b-ak}{abk}) + \frac{1}{2}(r)(\frac{1}{b}) = \frac{r}{b} \leq \frac{1}{k+1}
    \end{align}
    
\end{proofof}





\begin{theorem}{\thmlab{thm:ArbHamm}}
      \ThmHammingDistCorrectness
\end{theorem} 
\begin{proofof}{\thmref{thm:ArbHamm}}
We first note Authenticated Encryption security implies that the term $\eps_{AE}(\lambda)$ is negligible. Otherwise, an AE attacker could simply pick a random key $K'$ and use $c=\Enc_{K'}(m)$ as an attempted forgery for the unknown secret key $K$! 

 There are two conditions in the \defref{CondCorr} which need to be proved. The first condition is regular encryption correctness and the other one is the conditional encryption correctness. 

The observation that $\Dec_{sk}\left( \Enc_{pk}(m;r)\right) = m $ for all messages $m \in \Sigma^n$, random coins $r$ and all $(sk,pk)$ in the support of the key generation algorithm follows immediately from the correctness of Pallier encryption. 


It remains to to show that for all messages $ m_1, m_2 \in \Sigma^n$ such that $P_{\ell, \Ham}(m_1,m_2) $, all payload messages $m_3$, all $\{sk,pk\}$ in the support of our Key Generation algorithm and all random strings $ r_1, r_2 \in_R (\Mbb{Z}_{N}^*)^n$
we have


 
\begin{align}\eqnlab{EQ:EncCor2}	
\Pr \Biggl[\Dec_{sk}(\Cond\Enc_{pk}(c_1, m_2, m_3; r_2)) = m_3 \Biggl| \begin{matrix}
	(sk, pk)\gets \KG (1^\lambda) \\ 
	c_1 = \Enc_{pk}(m_1; r_1) \\
	P_{t,\Ham}  (m_1, m_2) = 1
	\end{matrix} \Biggl] \geq 1 - \eps \ .
  	\end{align}  
   
Let $K$ denote the authenticated encryption key and let  $ \ldb s \rdb_1,\ldots,   \ldb s \rdb_n$ denote the shares of $K$ that were generated by the conditional encryption algorithm. Let $\tilde{c} = (b,\tilde{c}_1,\ldots, \tilde{c}_n, C_{AE})$ denote the output of $\Cond\Enc_{pk}(c_1, m_2, m_3; r_2))$, and let  $\ldb s' \rdb_i = \RanDec\left( P.\Dec_{sk}(\tilde{c}_i) \right)$ denote the shares that are recovered. Finally, let $S^* = \{ i \in [n]~: ~m_2[i]=m_1[i]\}$ denote indices of the characters where $m_2$ and $m_1$ match. By correctness of Pallier we have $\ldb s' \rdb_i = \ldb s \rdb_i$  for {\em all} $i \in S^*$. For $i \not \in S^*$ the distribution over $\ldb s' \rdb_i$ is as follows: sample a uniformly random item $y_i$ from $\mathbb{Z}_N^*$ and output $y_i \mod{2^{\lambda}}$.




If $P_{Hamm,\ell}(m_1,m_2)=1$ we have $|S^*| \geq n-\ell$ and there is some subset $S \subseteq S^*$ of size $|S|=n-\ell$ such that \[ K=K_{S} = \recover\left( \left\{\left(i, \ldb s' \rdb_i\right)_{i \in S} \right\} \right)   \ . \]
From the correctness of the authenticated encryption scheme it follows that $\Auth.\Dec_{K_{S}}(c_{AE}) = m_3$.  

Thus, the only possible to output an incorrect message $m'$ is if for some $S \subseteq n$ of size $n-\ell$ we have $K \neq K_S = \recover\left( \left\{\left(i, \ldb s' \rdb_i\right)_{i \in S}\right\}\right)$ and $\Auth.\Dec_{K_{S^*}}(c_{AE}) \neq \bot$. However, if $K_S \neq K$ then $S \not \subseteq S^*$ and we can find some $i \in S \setminus S^* $. For now assume that for all $i \not\in S^*$ the value of $\ldb s' \rdb_i$ is uniformly random we can view $K_S$ as a uniformly random key. If we view each $K_S$ as random then we have $\Pr[\Auth.\Dec_{K_S}(c_{AE}) \neq \bot] \leq \epsilon_{AE}$ and $\Pr[\exists S \subseteq [n]~.  \Auth.\Dec_{K_S}(c_{AE}) \not\in \{m_3, \bot \} ] \leq {n \choose \ell} \eps_{AE}$. 

In the previous paragraph we assumed that the value $\ldb s' \rdb_i$ is uniformly random for each  $i \not\in S^*$ the value. This is close, but it is not quite true. In reality the distribution of $\ldb s' \rdb_i$ is described by sampling a uniformly random $y_i \in \mathbb{Z}_N^*$ and then outputting $y_i \mod{2^{\lambda}}$.However, by \thmref{thm:StatDistUak} the statistical distance between original/modified distribution of our recovered shares $ \ldb s' \rdb_1,\ldots,   \ldb s' \rdb_n$ is upper bounded by $2^{-\lambda}$. This follows since we are guaranteed that $N > n 2^{2\lambda}$ by definition of the key generation algorithm. Thus, we have 
\[ \Pr \Biggl[\Dec_{sk}(\Cond\Enc_{pk}(c_1, m_2, m_3; r_2)) \neq m_3 \Biggl| \begin{matrix}
	(sk, pk)\gets \KG (1^\lambda) \\ 
	c_1 = \Enc_{pk}(m_1; r_1) \\
	P_{t,\Ham}  (m_1, m_2) = 1
	\end{matrix} \Biggl] \leq  {n \choose \ell} \epsilon_{AE} + 2^{-\lambda} \ . \]

Similarly, if $P_{\ell,Hamm}(m_1,m_2)=0$ then for all $S \subseteq [n]$ of size $|S|=n-\ell$ we can (essentially) view $K_S$ as random since there is some $i \in S \setminus S^*$. It follows that 

\[ \Pr \Biggl[\Dec_{sk}(\Cond\Enc_{pk}(c_1, m_2, m_3; r_2)) \neq \bot \Biggl| \begin{matrix}
	(sk, pk)\gets \KG (1^\lambda) \\ 
	c_1 = \Enc_{pk}(m_1; r_1) \\
	P_{t,\Ham}  (m_1, m_2) = 0
	\end{matrix} \Biggl] \leq {n \choose \ell} \epsilon_{AE} + 2^{-\lambda} \ . \]




	
	
\end{proofof}


\begin{remindertheorem}{    \thmref{thm:CondSecArbHamm}}
 \thmSemiHonest
\end{remindertheorem} 
\begin{proofof}{\thmref{thm:CondSecArbHamm}}
	To prove this theorem we use a hybrid argument. In the first hybrid (Hybrid 0, real world) the distinguisher is given the actual ciphertext output conditional encryption and in the last hybrid contains the adversary is given a ciphertext output by our simulator --- described in \figref{fig:SimArbHamm}. As the hybrids are indistinguishable, we can conclude that the first and last hybrid are indistinguishable as well which implies that the our suggested construction is secure and provides conditional encryption secrecy in the semi-honest model. Then we concretely compute the distinguishing advantage of the defined hybrids. In what follows, we describe the hybrids with more details. 
	
	
	 	\begin{itemize}
	 	\item \textbf{Hybrid 0}: In this hybrid the distinguisher $ \D $ is given    $(sk, pk,  m_1, ,m_2,$ $ m_3, c_{m_1}, (1, \tilde{c}))$  in which $ \tilde{c} = (\tilde{c}_1, \dots, \tilde{c}_n, c_{AE})  \leftarrow \Cond\Enc_{pk}(c_1,m_2,m_3)$. 
	 	
	 	\item \textbf{Hybrid 1}: Let $T = \{i : m_2[i] \neq m_1[i]\}$ be the set of the indexes that $ m_1 $ and $ m_2 $ have different characters. We define \textbf{Hybrid 1} similar to \textbf{Hybird 0}, except for all $j\in T $ we replace $$\tilde{c}_{j}= P.\Enc_{pk}\Big(R_i(m_2[j]-m_1[j])+ \RanEnc(\ldb s \rdb_i)\Big)$$ with $P.\Enc(R_j’)$ where $ R_j' \in_R \Mbb{Z}_N$ are fresh and uniform random values chosen from $\Mbb{Z}_N$.  

	
	 	\item \textbf{Hybrid 2}: This hybrid is exactly the same as the previous hybrid except we replace all the remaining ciphertexts $ j\in [1:n]/ T  $ with $ \tilde{c}_{j}= P.\Enc_{pk}\Big(R_j(m_2[j]-m_1[j])+ \RanEnc(\ldb s_r \rdb_j)\Big) $ where $ \ldb s_r \rdb_j \in_R \{0,1\}^{\lambda} $ are fresh uniformly random elements (chosen independently from the secret $K$)  chosen from the field $\Mbb{F}_{2^\lambda}$.

           \item \textbf{Hybrid 3}: This hybrid is exactly the same as the previous hybrid except we replace all the ciphertexts $ j\in [1:n]/ T  $ with $ \tilde{c}_{j}= P.\Enc_{pk}\Big(R_j(m_2[j]-m_1[j])+ \hat{R}_j)\Big) $ where $ \hat{R}_j\in_R \Mbb{Z}_N $ are chosen from $\Mbb{Z}_{N}$ uniformly at random.

       \item \textbf{Hybrid 4}: This hybrid is exactly the same as the previous hybrid except we replace ciphertexts $ \tilde{c}_{j}$ for all $ j\in [1:n]/ T  $, with $\Pail.\Enc_{pk}(R'_j)$ in which $ R'_j\in_R \Mbb{Z}_N $ are chosen from $\Mbb{Z}_{N}$ uniformly at random.
           
	 	\item \textbf{Hybrid 5}: This hybrid is exactly the same as the previous hybrid unless we replace $ c_{AE} $ with $ c'_{AE} \in_R \{0,1\}^{l(\lambda)}$ a $\lambda$-bit string chosen uniformly at random. We note that $l(\lambda)$ is a polynomila over the security parameter $\lambda$ which represents the ciphertext size of authenticated encryption. 
	 
	 	\item \textbf{Hybrid 6}: We replace the ciphertext of the conditional encryption with the output of the simulator $ \Sim $ described in \figref{fig:SimArbHamm}.
	 	
	 	
	 	\end{itemize}
 	
 	
	Now we are proving that the defined hybrids are equivalent. 


\subsubsection{\textbf{Hybrid 0} $\equiv$ \textbf{Hybrid 1} } These hybrids are  equivalent i.e., we have 
	\begin{align}
& \Pr[\D^{H_{0}} =1] = \Pr[\D^{H_1} =1] \ . 
\end{align} 
Where $\D^{H_i}=1$ denotes the event that the distinguisher outputs $1$ in hybrid $i$. 
The argument is essentially the same as what we had for the security of \textit{Equality test} predicate --- see the proof of \thmref{thm:EqualityTestSecrecy}. In particular, for each $j \in T$ we have $m_1[j] \neq m_2[j]$ and $\left|m_1\left[j\right]-m_2\left[j\right]\right| \leq \min\{p,q\}$ which implies that $\left(m_1\left[j\right]-m_2\left[j\right]\right) \in \mathbb{Z}_N^*$. It follows that $R_j \times \left(m_1\left[j\right]-m_2\left[j\right]\right)$ is uniformly random in $\mathbb{Z}_N$.  


\subsubsection{\textbf{Hybrid 1} $ \equiv $ \textbf{Hybrid 2}} We have information theoretically eliminated all information about shares $\shrs{}$ with $j \in T$. Since $P_{\ell,\Ham}(m_1,m_2)=0$ we have $|T| > \ell$ and $|\overline{T}| < n-\ell$. Let $T=\{i_1,\ldots, i_t\}$ with $t < n-\ell$. Shamir Secret Sharing guarantees that $(s_{i_1} , s_{i_2} , \ldots , s_{i_t})$ is uniformly random in $\mathbb{F}_{2^{\lambda}}^t$. Thus, we can simply replace the shares with uniformly random values. We have \begin{align}
& \Pr[\D^{H_{1}} ] =   \Pr[\D^{H_2} ] \ . 
\end{align} 



\subsubsection{Statistically indistinguishability of \textbf{Hybrid 2} $ \equiv $ \textbf{Hybrid 3}} We apply \thmref{thm:StatDistUak} with $a=2^{\lambda}$, $k = \lfloor \frac{N}{2^\lambda} \rfloor$ and $b=N$. We first observe that when $i \in \overline{T}$ the value of $s_i \in \mathbb{F}_{2^{\lambda}}$ is uniformly random so that $\RanEnc(s_i)$ is equivalent to $\mathcal{D}_{ak}$. It follows that the statistical distance between $\RanEnc(s_i)$ and the uniform ditribution $\Mbb{Z}_N$ is at most $\frac{1}{k} = \lfloor \frac{N}{2^\lambda} \rfloor^{-1}$. Since we are replacing the random value in $|\overline{T}|$ ciphertexts the overall statistical distance is upper bounded by $\frac{|\overline{T}|}{k} \leq \frac{n}{k}$  we have: 

\begin{align}
	& |\Pr[\D^{H_{2}} ] -   \Pr[\D^{H_3}]| \leq \frac{2^\lambda n}{N-2^{\lambda}} \leq 2^{-\lambda} \ . 
\end{align}
The last inequality follows since  we pick $N \geq 2n 2^{2\lambda}$ so that $\frac{2^\lambda n}{N-2^{\lambda}} \leq 2^{-\lambda}$.   


\subsubsection{\textbf{Hybrid 3} $\equiv$ \textbf{Hybrid 4}} These hybrids are statistically indistinguishable as $R_j (m_2[j]-m_1[j]) + \hat{R}_j$ is already uniformly random in $\mathbb{Z}_N$. We have 
\begin{align}
	& \Pr[\D^{H_{3}} ] = \Pr[\D^{H_4} ] 
\end{align}



\subsubsection{Indistinguishability of  \textbf{Hybrid 4} and \textbf{Hybrid 5}}  By Hybrid 4 we have information theoretically elimated any information about the secret key $K$ for our authentication encryption scheme from $(\tilde{c}_1,\ldots, \tilde{c}_n)$. Thus, by AE security any adversary running in time at most $t_{AE} = t_{AE}(\lambda)$ can distinguish between $c_{AE}$ and $c_{AE}'$ with the advantage of at most $\eps_{AE}(t_{AE}, \lambda)$. So we have 

	\begin{align}
	& |\Pr[\D^{H_{4}} ] -   \Pr[\D^{H_1}=1]|  \leq \eps_{AE}(t_{AE}, \lambda)
	\end{align}


\subsubsection{\textbf{Hybrid 5}  $\equiv $ \textbf{Hybrid 6}} Looking at the definition of our our simulator in \figref{fig:SimArbHamm}, we can see that the conditionally encrypted ciphertext in Hybrids 5 and 6 are generated in exactly the same way. It follows that the hybrids are 
information-theoretically equivalent and we have 
\begin{align}
	& \Pr[\D^{H_{5}}] = \Pr[\D^{H_6} ] 
\end{align}

Putting everything together we have 
\begin{align}
&\Big| \Pr\left[\D \left(sk, pk, m_1, m_2, m_3. \Cond\Enc_{pk}\left(\Enc_{pk}\left(m_1\right), m_2\right)\right)=1\right] \nonumber\\
&-  \Pr\left[\D \left(sk, pk, m_1, m_2, m'_3,\Sim\left(pk\right)\right)=1\right]\Big| \nonumber \\
&= \left|\Pr\left[\D^{H_0}\right]-\Pr\left[\D^{H_6}\right]\right| \nonumber \\
&\leq \sum_{i=0}^5 \left|\Pr\left[\D^{H_i}\right]-\Pr\left[\D^{H_{i+1}}\right]\right| \nonumber \\
&< \eps_{AE}(t',\lambda) + \frac{n2^\lambda}{N-2^{\lambda}} \leq \eps_{AE}(t',\lambda) + 2^{-\lambda}  \nonumber \ .
\end{align}

\begin{figure*}
		\begin{itemize}
			\item [] \underline{Design of simulator $ \Sim(pk) $}
			\begin{itemize}
				\item [1.] Sample, $ r''_1, \ldots, r''_n \in_R\Mbb{Z}^*_N, R''_1, \ldots, R''_n \in_R{\Mbb{Z}_N}^{n} $ uniformly at random  
    
				\item [2.] For all $ 1\leq i \leq n $ compute $ \tilde{c}'_i = \Pail.\Enc_{pk}(R''_i; r''_i ) $
				
				\item [3.] Pick $ R_K\in_R \set{0,1}^{l(\lambda)} $ uniformly at random and set $ c'_{AE} = R_K $.  //{\color{blue} $l(\lambda)$ represents the ciphertext size of our authenticated encryption.}

				\item [4.] Output $ \tilde{c}' = (1, \tilde{c}'_1, \cdots, \tilde{c}'_n,  c'_{AE}) $. 
			\end{itemize}
		\end{itemize}\caption{Steps of designing the simulator $ \Sim $ for the conditional encryption secrecy when the predicate is $P_{\ell, \Ham}$}
		\figlab{fig:SimArbHamm}
\end{figure*}

\end{proofof}


\begin{remindertheorem} {\thmref{thm:EDCor}}
\thmEDCorrect
\end{remindertheorem}

\begin{proofof}{\thmref{thm:EDCor}}
Note that $\Enc_{pk}(m)$ includes $c[0] = P.\Enc_{pk}(\ToInt(m))$ and that therefore by correctness of Pallier we have  $\Dec_{sk}\left( \Enc_{pk}(m)\right) = \Dec_{sk}\left( P.\Enc_{pk}(\ToInt(m))\right) = \ToOrig$ $(\ToInt(m)) = m$ with probability $1$ for all messages $m \in \Sigma^{\leq n}$ and all public/private key pairs in the support of $\KG$. 

Recall that if $c = (0,c[0],\ldots, c[n])=\Enc_{pk}(m)$ then $\Cond\Enc_{pk}(c,m',m'')$ will output a ciphertext of the form $(1,\tilde{c}_0,\ldots,\tilde{c}_{2n})$. If $P_{1,\ED}(m,m') = 0$ then we have $m_{-j} \neq m'$ and $m \neq m'_{-j}$ for all $0 \leq j \leq n$. Thus, by \thmref{thm:EqualityTestSecrecy} each $\tilde{c}_j = g^{y_j} r_j^n \mod{N^2}$ for random values $r_j \in \mathbb{Z}_N^*$ and $y_j \in \mathbb{Z}_N$. Thus, we have \[\Pr[\Dec_{sk}(1,\tilde{c}_0,\ldots,\tilde{c}_{2n}) \neq \bot] \leq \Pr[\exists j. y_j < |\Sigma|^{n+1}] \leq \frac{(2n+1) |\Sigma|^{n+1}}{N} \ .\] This implies that the construction is $1-\epsilon(\lambda)$-error detecting conditional encryption scheme with $\eps(\lambda) = \frac{|\Sigma|^{n+1}}{N} \leq \frac{1}{\max\{p,q\}}$.

Finally, if $P_{1,\ED}(m,m') = 0$ then by perfect correctness of our conditional encryption scheme for equality predicate there exists some $j$ such that $\tilde{c}_j = g^{y_j} r_j^N \mod{N^2}$ is a valid Pallier encryption of  $y_j=\ToInt(m'')$. Furthermore, we have already shown that $\Pr[\exists j. y_j < |\Sigma|^{n+1} \wedge y_j \neq \ToInt(m'')] \leq \frac{(2n+1) |\Sigma|^{n+1}}{N}$. It follows that, except with probability $\frac{(2n+1) |\Sigma|^{n+1}}{N}$ that we will have $\Dec_{sk}(1,\tilde{c}_0,\ldots,\tilde{c}_{2n}) = \ToOrig(\ToInt(m''))$. 
    
\end{proofof}

\begin{remindertheorem}{\thmref{ORComp:Security}}\thmORPrivacy
\end{remindertheorem}
\begin{proofof}{\thmref{ORComp:Security}}
    The simulator $\Sim_{OR}(pk)$ for $\Pi_{OR}$ will run the simulator $\Sim_i(pk_i)$ for each conditional encryption scheme \footnote{In the malicious security setting the simulator $\Sim_{OR}(b, pk)$ is also given a bit $b=1$ if and only if $\Cond\Enc_{pk}(c,m',m'')=\bot$ i.e., if and only if $\Pi_i.\Cond\Enc_{pk}(c,m',m'')=\bot$ for some $i \leq k$. If $b=1$ then $\Sim_{OR}(b,pk)$ outputs $\bot$. Otherwise we simply run $\Sim_i(0,pk_i)$ for each $i \leq k$.} and concatenate all of the ciphertexts. Clearly, the running time of the simulator is $t_{\Sim}'(\lambda) \approx \sum_{i=1}^k t_{\Sim,i}(\lambda)$. We can now define a sequence of $k+1$ hybrids Hybrid $0$ to Hybrid $k$. Intuitively, in hybrid $i$ we set $c_j = \Sim_i(pk)$ for $j\leq i$ and $c_j = \Pi_j.\Cond\Enc(c,m',m'')$ for $j > i$. Note that in Hybrid $0$ we have $c_j = \Pi_j.\Cond\Enc(c,m',m'')$ for all $j$ and thus the final output is $\Cond\Enc(c,m',m'')$. By contrast, in Hybrid $k$ we have   $c_j = \Sim_j(pk_j)$ for all $j \leq k$ and thus the final output is $\Sim_{OR}(pk)$. 
        
    By assumption any attacker running in time $t'(\lambda)=t(\lambda) - o(t(\lambda))$ can distinguish hybrids $i-1$ and $i$ with probability at most $\epsilon_i(t(\lambda), \lambda)$. It follows that any attacker running in time $t'(\lambda)=t(\lambda)  - o(t(\lambda))$ can  distinguish hybrid $0$ from hybrid $k$ with probability at most  $\epsilon'(\lambda,t'(\lambda)) = \sum_{i=1}^k  \epsilon_i(t(\lambda), \lambda)$. 
\end{proofof}

\begin{remindertheorem}{\thmref{ORComp:Correct}}
    \thmORCorrect
\end{remindertheorem}
\begin{proofof}{\thmref{ORComp:Correct}}
Let $T= \{j: P_j(m_1,m_2)=1\}$ and $\overline{T} = \{j: P_j(m_1,m_2)=0\} = [k] \setminus T$.  We first suppose that $P_{OR}(m_1,m_2)=0$ which implies that $P_i(m_1,m_2)=0$ for all $i \leq k$ i.e., $\overline{T} = [k]$. 

Now let $(pk,sk)$ be any honestly generated public/secret key and let $c = (0,c_1,\ldots,c_k) = \Cond\Enc_{pk}(m_1)$ with $c_i \doteq \Pi_i.\Enc_{pk}(m_1)$. The probability that $\Pi_i.\Dec_{sk}\left(\Pi_i.\Cond\Enc_{pk}(c_i,m_2,m_3)\right) \neq \bot$ is at most $\epsilon'_i(\lambda)$. Union bounding over all $i \leq k$ the probability that there exists $i$ such that $\Pi_i.\Dec_{sk}\left(\Pi_i.\Cond\Enc_{pk}(c_i)\right) \neq \bot$ is at most $\sum_{i=1}^k \epsilon'_i(\lambda) = \epsilon'(\lambda)$.

On the other hand suppose that $P_{OR}(m_1,m_2)=1$ which means that $P_j(m_1,m_2)=1$ for some $j\leq k$. Clearly, if $|T|\geq 1$ and $\Pi_i.\Dec_{sk}\left(\Pi_i.\Cond\Enc_{pk}(c_i)\right) = m_3$ for all $i \in T$ and $\Pi_i.\Dec_{sk}\left(\Pi_i.\Cond\Enc_{pk}(c_i)\right) = \bot$ for all $i \not \in T$ then $\Dec_{sk}$ will output the correct message $m_3$. As before the probability that there exists $i \in \overline{T}$ such that $\Pi_i.\Dec_{sk}\left(\Pi_i.\Cond\Enc_{pk}(c_i)\right) \neq \bot$ is at most $\sum_{i=1}^k \epsilon'_i(\lambda) = \epsilon'(\lambda)$. Similarly, the probability that there exists $j \in T$ such that $\Pi_i.\Dec_{sk}\left(\Pi_i.\Cond\Enc_{pk}(c_i)\right) \neq m_3$ is at most $\sum_{j=1}^k \epsilon_i(\lambda)$. 

Thus, we have $\Pr\left[\Dec_{sk}\left(\Cond\Enc_{pk}\left( \Enc_{pk}(m_1) ,m_2,m_3 \right)\right) \neq m_3 \right] \leq \sum_{i=1}^k \left( \epsilon_i'(\lambda)+\epsilon_i(\lambda)\right) = \epsilon(\lambda)$.



\end{proofof}
\input{systemmodelTypTop}
\input{securityDefTypTop}
\input{TyptopConst}
\input{ConstructionsFigures}
\input{ArtifactAppendix}

\end{document}
\endinput
%%
%% End of file `sample-sigconf.tex'.
