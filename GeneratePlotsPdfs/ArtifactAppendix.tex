\section{Artifact Appendix}

\emph{The Artifact Appendix is meant to summarize the purpose,
availability, content, dependencies, and execution of the artifacts
that support your paper.  It should include a clear description of
your artifacts' hardware, software, and configuration requirements.
If your artifacts have received the \emph{Artifacts
Evaluated--Functional}, \emph{Artifacts Evaluated--Reusable}, or
\emph{Results Reproduced} badge, this appendix should summarize the
relevant major claims made by your paper and provide instructions for
validating each claim through the use of your artifacts.  Linking the
claims of your paper to the artifacts allows a reader of your paper
to more easily try to reproduce your paper's results.}

\emph{Please fill in all the mandatory sections, keeping their titles
and organization but removing the current illustrative content, and
remove the optional sections that do not apply to your artifacts.}

%%%%%%%%%%%%%%%%%%%%%%%%%%%%%%%%%%%%%%%%%%%%%%%%%%%%%%%%%%%%%%%%%%%%%

\subsection{Abstract}

\emph{[Mandatory]}
%
\emph{Provide a short description of your artifacts.  What do they
contain?  How are they packaged?  What do they do when executed?  What
results to they produce?  How are your artifacts licensed?  Be brief;
subsequent sections of this appendix can be used fill in the details.}

%%%%%%%%%%%%%%%%%%%%%%%%%%%%%%%%%%%%%%%%%%%%%%%%%%%%%%%%%%%%%%%%%%%%%

\subsection{Description \& Requirements}

\emph{This section should further detail your artifacts and summarize
the requirements that must be met for a person to recreate an
experimental environment for utilizing your artifacts.  Where
applicable, state the minimum hardware and software requirements for
running your artifacts.  It is also good practice to list and describe
in this section benchmarks where those are part of, or simply have
been used to produce results with, your artifacts.}

\emph{[Mandatory]}
%
\emph{Here, provide additional details about the general organization
of your artifacts.  For example, provide a general roadmap to the
files within your artifacts.  Where is the \texttt{README}?  Where are
the source files, the datasets, the scripts for running experiments,
etc.?  You don't need to describe every file and directory here; just
provide a general ``lay of the land.''}

%%%%%

\subsubsection{Security, privacy, and ethical concerns}

\emph{[Mandatory]}
%
\emph{Describe any risks that people might be exposed to while
executing your artifacts: e.g., risks to their machines' security,
data privacy, or other ethical concerns.  This is particularly
important if destructive steps are taken or security mechanisms are
disabled during execution.}

%%%%%

\subsubsection{How to access}

\emph{[Mandatory]}
%
\emph{Describe how to access your artifacts.  If your artifacts have
received the \emph{Artifacts Available} badge, you must provide the
DOI(s) for the permanently and publicly available copies of your
artifacts.  Most likely, this is the version of your artifacts that
you deposited with Zenodo.}

\emph{You may describe more than one way to access your artifacts.
For example, if the archived versions of your artifacts are available
at Zenodo, and you are also making maintained versions of your
artifacts available through GitHub, you can describe how to access
both versions of your artifacts.}

%%%%%

\subsubsection{Hardware dependencies}

\emph{[Mandatory]}
%
\emph{State any specific hardware features that are required to make
use of your artifacts: e.g., vendor, CPU/GPU/FPGA, number of
processors/cores, microarchitecture, interconnect, memory, hardware
counters, etc.  If your artifacts do not have significant hardware
dependencies, simply write ``None'' in this section.}

%%%%%

\subsubsection{Software dependencies}

\emph{[Mandatory]}
%
\emph{State any specific OS and software packages that are required to
make use of your artifacts.  This is particularly important if you
share your source code and it must be compiled, or if your artifacts
rely on proprietary software that is not included in your artifact
packages.  If your artifacts do not have significant software
dependencies, simply write ``None'' in this section.}

%%%%%

\subsubsection{Benchmarks}

\emph{[Mandatory]}
%
\emph{Describe any data (e.g., datasets, models, workloads,
etc.)\ required by the experiments that are reported in your paper and
supported by your artifacts.  If this does not apply to your
artifacts, simply write ``None'' in this section.}

%%%%%%%%%%%%%%%%%%%%%%%%%%%%%%%%%%%%%%%%%%%%%%%%%%%%%%%%%%%%%%%%%%%%%

\subsection{Set Up}

\emph{This section should summarize the installation and configuration
steps that a person should follow to prepare an environment for making
use of your artifacts.  If the set-up steps are complicated, point the
reader to the places in your artifacts where detailed set-up
instructions can be found.}

%%%%%

\subsubsection{Installation}

\emph{[Mandatory]}
%
\emph{Provide instructions for downloading and installing dependencies
as well as the main artifacts.  After following these steps, a user of
your artifacts should be able to run a simple functionality test.}

%%%%%

\subsubsection{Basic test}

\emph{[Mandatory]}
%
\emph{Provide instructions for running a simple functionality test.
This test does not need to exercise all the features of your
artifacts, but ideally, it should allow a user to check that all of
the required software components are correctly installed and
functioning as intended.  Please include the expected successful
output and any required input parameters.}

%%%%%%%%%%%%%%%%%%%%%%%%%%%%%%%%%%%%%%%%%%%%%%%%%%%%%%%%%%%%%%%%%%%%%

\subsection{Evaluation Workflow}

\emph{This section should include all the operational steps and
experiments that a person must be carry out to utilize in your
artifacts toward the goal of validating your paper's key results and
claims.  To that end, we ask you to use the two following subsections
and cross-reference the items therein as explained below.}

%%%%%

\subsubsection{Major claims}

\emph{[Mandatory for \emph{Artifacts Evaluated--Functional},
  \emph{Artifacts Evaluated--Reusable}, and \emph{Results Reproduced};
  optional for \emph{Artifacts Available}]}
%
\emph{Enumerate here the major claims (Cx) made in your paper.  Follow
the examples below.}
\bigskip

\begin{compactitem}
\item[(C1):]
  %
  \emph{System\_name achieves the same accuracy as the
  state-of-the-art systems for a task X while requiring Y\% less
  storage.  This is proven by the experiment (E1) described in [refer
    to your paper's sections], whose results are illustrated/reported
  in [refer to your paper's plots, tables, sections, etc.].}

\item[(C2):]
  %
  \emph{System\_name has been used to uncover new bugs in the X
  software.  This is proven by experiments (E2) and (E3) in [refer to
    your paper's sections].}
\end{compactitem}

%%%%%

\subsubsection{Experiments}

\emph{[Mandatory for \emph{Artifacts Evaluated--Functional},
  \emph{Artifacts Evaluated--Reusable}, and \emph{Results Reproduced};
  optional for \emph{Artifacts Available}]}
%
\emph{Explicitly link the descriptions of your experiments to the
items you have provided in the previous subsection about major claims.
Please provide estimates of the person- and compute-time required for
each of the listed experiments (using the suggested hardware/software
configuration).  Follow the examples below.}

\bigskip
\begin{compactitem}
\item[(E1):]
  %
  \emph{[Optional Name] [30~person-minutes + 1~compute-hour + 5\,GB
    disk]: Provide a short explanation of the experiment and expected
  results.}

  \emph{Provide the steps to perform the experiment and collect and
  organize the results as expected from your paper.  We encourage you
  to use the following structure with three main blocks for the
  description of your experiment.  If the procedures are complicated,
  point the reader to the places in your artifacts where the full
  instructions can be found.}

  \begin{asparadesc}
  \item[Preparation:] \emph{Describe the steps required to prepare and
  configure the environment for this experiment.}

  \item[Execution:] \emph{Describe the steps required to run this
  experiment.}

  \item[Results:] \emph{Describe the steps required to collect and
  interpret the results for this experiment.}
  \end{asparadesc}

\item[(E2):] \emph{[Optional Name] [1~person-hour + 3~compute-hours]: \ldots}

\item[(E3):] \emph{[Optional Name] [1~person-hour + 3~compute-hours]: \ldots}
\end{compactitem}
\bigskip

\emph{In all of the above blocks, please provide indications about the
expected outcome for each of the steps (given the suggested hardware
and software configuration).}

%%%%%%%%%%%%%%%%%%%%%%%%%%%%%%%%%%%%%%%%%%%%%%%%%%%%%%%%%%%%%%%%%%%%%

\subsection{Notes on Reusability}

\emph{[Optional]}
%
\emph{This section is meant to provide additional information about
how a person could utilize your artifacts beyond the research
presented in your paper.  A broad objective of artifact evaluation is
to encourage you to make your research artifacts reusable by others.}

\emph{Include in this section any instructions, guidance, or advice
that you believe would help others reuse your artifacts.  For example,
you could describe how one might scale certain components of your
artifacts up or down; apply your artifacts to different kinds of
inputs or datasets; customize your artifacts' behavior by replacing
specific modules or algorithms; etc.}

%%%%%%%%%%%%%%%%%%%%%%%%%%%%%%%%%%%%%%%%%%%%%%%%%%%%%%%%%%%%%%%%%%%%%

\subsection{Version}
%%%%%%%%%%%%%%%%%%%%
% Obligatory.
% Do not change/remove.
%%%%%%%%%%%%%%%%%%%%
Based on the LaTeX template for Artifact Evaluation V20220926.

%%%%%%%%%%%%%%%%%%%%%%%%%%%%%%%%%%%%%%%%%%%%%%%%%%%%%%%%%%%%%%%%%%%%%

%% AUTHORSHIP AND PROVENANCE OF THIS TEMPLATE

%% This template was created by Eric Eide for the ACM CCS 2024
%% conference.
%%
%% The AEC Chair for CCS 2024 was:
%% Eric Eide (University of Utah)

%% This template is based on the "USENIX Security 2023 AE Template."
%% https://secartifacts.github.io/usenixsec2023/appendix/usesec23-ae-latex.zip
%%
%% The AEC Co-chairs for USENIX Security 2023 were:
%% Cristiano Giuffrida (Vrije Universiteit Amsterdam)
%% Anjo Vahldiek-Oberwagner (Intel Labs)

%% The USENIX Security AE template was based on the
%% % Artifact Appendix Template for EuroSys'22 AE
%% https://sysartifacts.github.io/eurosys2022/appendix/EuroSys22_ArtifactAppendix_template.tex
%%
%% The AEC Co-chairs for EuroSys 2022 were:
%% Natacha Crooks (UC Berkeley)
%% Solal Pirelli (EPFL)
%% Salvatore Signorello (University of Lisbon)
%% Anjo Vahldiek-Oberwagner (Intel Labs)

%% The above were based on:

%% LaTeX template for Artifact Evaluation V20220119
%%
%% Original Authors
%% * Grigori Fursin (cTuning foundation, France) 2014-2020
%% * Bruce Childers (University of Pittsburgh, USA) 2014
%%
%% Modified by
%% * Clementine Maurice (CNRS, France) 2021-2022
%% * Cristiano Giuffrida (Vrije Universiteit Amsterdam, Netherlands) 2021-2022
%%
%% (C) Copyright 2014-2022
%%
%% CC BY 4.0 license
%%