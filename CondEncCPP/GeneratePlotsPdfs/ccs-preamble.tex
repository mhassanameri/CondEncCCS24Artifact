%\usepackage[protrusion=true,expansion=true]{microtype}
%\usepackage{makeidx}

%\usepackage[addedmarkup=uline,deletedmarkup=sout]{changes}
% \usepackage{changes}

% \usepackage{cite}
\usepackage{hyperref}
%Tracking the changes: 

%Fixing the special symbols in the bib files and their corresponding doi for ``splncs04'' bib style. 
\makeatletter
\AtBeginDocument{%
  \def\doi#1{\url{https://doi.org/#1}}}
\makeatother
\usepackage{textcomp} 

%For reducing the free spaces between the sections (This works for reducing the free spaces).
\usepackage{parskip} 
% \usepackage{dblfloatfix}
\usepackage{multicol}
\usepackage{prot}
\usepackage{amsmath}
%\usepackage{breqn}
\let\Bbbk\relax
\usepackage{amssymb,amsfonts}
\usepackage{latexsym}
\usepackage{subcaption}
\usepackage{graphicx}
\usepackage[export]{adjustbox}
\usepackage{fancybox}
\usepackage{floatrow}
\usepackage{tikz}
\setlength{\abovecaptionskip}{-5in} 
\usepackage{tikzit}
\input{1.tikzstyles}
\usepackage{makecell}
\usepackage{lipsum,adjustbox}

\usepackage{pgfplots}
% \usepackage{xintexpr}
%\usepackage[active,tightpage]{preview}
%\PreviewEnvironment{tikzpicture}
\pgfplotsset{compat=newest}
\usetikzlibrary{decorations.pathreplacing}
\usepackage{setspace}
\usepackage{algorithm}
\usepackage[noend]{algpseudocode}
\usepackage[framemethod=tikz]{mdframed}
\usepackage{xspace}
% \usepackage{pgfplots}
\usepackage{framed}
\usepackage{float}
% \pgfplotsset{compat=3}
\usepackage{tabularx}

\setcounter{secnumdepth}{4}

\usepackage{xintexpr}
% \usepackage[active,tightpage]{preview}
\usepackage[tightpage]{preview}

\raggedbottom

\let\llncssubparagraph\subparagraph
%% Provide a definition to \subparagraph to keep titlesec happy
\let\subparagraph\paragraph
% \usepackage{titlesec}
% \titlespacing*{\subsubsection}{0pt}{1ex}{1ex}

\PreviewEnvironment{tikzpicture}
 %\newenvironment{proof}{\noindent{\bf Proof : \ }}{\hfill$\Box$\par\medskip}

%\newenvironment{proof}{\noindent{\bf Proof : \ }}{\hfill$\Box$\par\medskip}

%\newenvironment{construction}{\noindent{}}{}
\newenvironment{proofhack}{\noindent {\bf Proof}.\ }{}
%\newtheorem{theorem}{Theorem}[section]
%\newtheorem{corollary}[theorem]{Corollary}
%\newtheorem{lemma}[theorem]{Lemma}
%\newtheorem{proposition}[theorem]{Proposition}
%\newtheorem{definition}[theorem]{Definition}
%\newtheorem{experiment}[theorem]{Experiment}
%\newtheorem{remark}[theorem]{Remark}
%\newtheorem{claim}[theorem]{Claim}
%\newtheorem{invariant}[theorem]{Invariant}
%\newtheorem{observation}[theorem]{Observation}
%%\newtheorem{example}[theorem]{Example}
%\newtheorem{fact}[theorem]{Fact}
%%\newtheorem{conjecture}[theorem]{Conjecture}
%\newtheorem{problem}[theorem]{Problem}
%\newtheorem{construction}[theorem]{Contruction}
%%\newtheorem{property}[theorem]{Property}

\newenvironment{proofof}[1]{\begin{trivlist} \item {\bf Proof
#1:~~}}
  {\qed\end{trivlist}}
\renewenvironment{proofof}[1]{\par\medskip\noindent{\bf Proof of #1: \ }}{\hfill$\Box$\par\medskip}
\newcommand{\namedref}[2]{\hyperref[#2]{#1~\ref*{#2}}}
\newcommand{\thmlab}[1]{\label{thm:#1}}
\newcommand{\thmref}[1]{\namedref{Theorem}{thm:#1}}
\newcommand{\lemlab}[1]{\label{lem:#1}}
\newcommand{\lemref}[1]{\namedref{Lemma}{lem:#1}}
\newcommand{\claimlab}[1]{\label{claim:#1}}
\newcommand{\claimref}[1]{\namedref{Claim}{claim:#1}}
\newcommand{\corlab}[1]{\label{cor:#1}}
\newcommand{\corref}[1]{\namedref{Corollary}{cor:#1}}
\newcommand{\seclab}[1]{\label{sec:#1}}
\newcommand{\secref}[1]{\namedref{Section}{sec:#1}}
\newcommand{\applab}[1]{\label{app:#1}}
\newcommand{\appref}[1]{\namedref{Appendix}{app:#1}}
\newcommand{\factlab}[1]{\label{fact:#1}}
\newcommand{\factref}[1]{\namedref{Fact}{fact:#1}}
\newcommand{\remlab}[1]{\label{rem:#1}}
\newcommand{\remref}[1]{\namedref{Remark}{rem:#1}}
\newcommand{\figlab}[1]{\label{fig:#1}}
\newcommand{\figref}[1]{\namedref{Figure}{fig:#1}}
\newcommand{\alglab}[1]{\label{alg:#1}}
\renewcommand{\algref}[1]{\namedref{Algorithm}{alg:#1}}
\newcommand{\tablelab}[1]{\label{tab:#1}}
\newcommand{\tableref}[1]{\namedref{Table}{tab:#1}}
\newcommand{\deflab}[1]{\label{def:#1}}
\newcommand{\defref}[1]{\namedref{Definition}{def:#1}}
\newcommand{\explab}[1]{\label{exp:#1}}
\newcommand{\expref}[1]{\namedref{Experiment}{exp:#1}}
\newcommand{\eqnlab}[1]{\label{eq:#1}}
\newcommand{\eqnref}[1]{\namedref{Equation}{eq:#1}}
\newcommand{\constref}[1]{\namedref{\textbf{Construction}}{const:#1}}
\newcommand{\constlab}[1]{\label{const:#1}}
\newcommand{\obslab}[1]{\label{obs:#1}}
\newcommand{\obsref}[1]{\namedref{Observation}{obs:#1}}
\newcommand{\exlab}[1]{\label{ex:#1}}
\newcommand{\exref}[1]{\namedref{Example}{ex:#1}}
\newcommand{\stepref}[1]{\namedref{Step}{step:#1}}
\newcommand{\steplab}[1]{\label{step:#1}}
\newcommand{\propref}[1]{\namedref{Property}{prop:#1}}
\newcommand{\proplab}[1]{\label{prop:#1}}
\newcommand{\probref}[1]{\namedref{Problem}{prob:#1}}
\newcommand{\problab}[1]{\label{prob:#1}}
\newcommand{\conjref}[1]{\namedref{Conjecture}{conj:#1}}
\newcommand{\conjlab}[1]{\label{conj:#1}}
%\newcommand{\propref}[1]{\namedref{Property}{prop:#1}}
%\newcommand{\proplab}[1]{\label{prop:#1}}


\newenvironment{remindertheorem}[1]{\medskip \noindent {\bf Reminder of  #1.  }\em}{}


\def \aadapt    {\mdef{\mathsf{A}_{adaptive}}}
\def \asingle    {\mdef{\mathsf{A}_{single}}}
\def \HAM    {\mdef{\mathsf{HAM}}}
\def \time    {\mdef{\mathsf{time}}}
\def \superconc    {\mdef{\mathsf{superconc}}}
\def \grates    {\mdef{\mathsf{grates}}}
\def \depth    {\mdef{\mathsf{depth}}}
\def \scrypt    {\mdef{\mathsf{scrypt}}}
\def \sinks    {\mdef{\mathsf{sinks}}}
\def \indeg    {\mdef{\mathsf{indeg}}}
\def \parents    {\mdef{\mathsf{parents}}}
\def \ancestors    {\mdef{\mathsf{ancestors}}}
\def \lab    {\mdef{\mathsf{lab}}}
\def \unique    {\mdef{\mathsf{UNIQUE}}}
\def \potential    {\mdef{\mathsf{PotentialParents}}}
\def \partition    {\mdef{\mathsf{BlockPartition}}}
\def \crpartition    {\mdef{\mathsf{CR-BlockPartition}}}
\def \cruniform    {\mdef{\mathsf{CR-Uniform}}}
\def \A    {\mdef{\mathcal{A}}}
\def \D    {\mdef{\mathcal{D}}}
\def \C    {\mdef{\mathcal{C}}}
\def \B    {\mdef{\mathcal{B}}}
\def \M   {\mdef{\mathcal{M}}}
\def \U   {\mdef{\mathcal{U}}}
\def \S   {\mdef{\mathcal{S}}}
\def \R   {\mdef{\mathcal{R}}}
\def \d    {\mdef{\delta}}
\def \q    {\mdef{\mathbf{q}}}
\def \G    {\mdef{\mathbb{G}}}
\def \H    {\mdef{\mathcal{H}}}
\def \MHF    {\mdef{\mathsf{MHF}}}
\def \cmc    {\mdef{\mathsf{cmc}}}
\def \Enc    {\mdef{\mathsf{Enc}}}
\def \Dec    {\mdef{\mathsf{Dec}}}
\def \FHE    {\mdef{\mathsf{FHE}}}

\def \Cond   {\mdef{\mathsf{C}}}
\def \KG   {\mdef{\mathsf{KeyGen}}}
\def \Setup   {\mdef{\mathsf{Setup}}}
\def \Eval   {\mdef{\mathsf{Eval}}}

\def \Prove   {\mdef{\mathsf{Prove}}}
\def \Verify   {\mdef{\mathsf{Verify}}}
\def \Validate   {\mdef{\mathsf{Validate}}}
\def \PKE   {\mdef{\mathsf{PKE}}}
\def \iO {\mdef{i\mathcal{O}}}
\def \GoodKey {\mdef{\mathtt{GOODKEYS}}}


\def \Sim   {\mdef{\mathtt{Sim}}}
\def \CE   {\mdef{\mathtt{CE}}}
\def \AE   {\mdef{\mathtt{AE}}}
\def \PKDF   {\mdef{\mathtt{PKDF}}}
\def \Derive   {\mdef{\mathtt{Derive}}}

\def \lp    {\mdef{\mathsf{lp}}}
\def \pp    {\mdef{\mathsf{pp}}}
\def \LP    {\mdef{\mathsf{LP}}}
\def \bad    {\mdef{\mathsf{bad}}}
\def \chal    {\mdef{\mathsf{chal}}}
\def \sparse    {\mdef{\mathsf{sparse}}}
\def \Setup    {\mdef{\mathsf{Setup}}}
\def \miss    {\mdef{\mathsf{MISS}}}
\def \costly    {\mdef{\mathsf{COSTLY}}}
\def \setup    {\mdef{\textbf{setup}}}
\def \challenge    {\mdef{\textbf{challenge}}}
\def \query    {\mdef{\textbf{query}}}
\def \guess    {\mdef{\textbf{guess}}}
\def \mhfeval  {\mdef{\mathsf{MHF.Eval}}}
\def \etrace    {\mdef{\mathsf{etrace}}}
\def \valiant    {\mdef{\mathsf{Valiant}}}
\def \request    {\mdef{\mathsf{request}}}
\def \store    {\mdef{\mathsf{store}}}
\def \load    {\mdef{\mathsf{load}}}
\def \True    {\mdef{\mathsf{True}}}
\def \False    {\mdef{\mathsf{False}}}
\def \invariant    {\mdef{\mathsf{invariant}}}
\def \permute    {\mdef{\mathsf{permutation}}}
\def \counter    {\mdef{\mathsf{counter}}}
\def \rmerge    {\mdef{\mathsf{RandomMerge}}}
\def \Write    {\mdef{\mathsf{Write}}}
\def \Read    {\mdef{\mathsf{Read}}}
\def \overlay    {\mdef{\mathsf{overlay}}}
\def \SC    {\mdef{\mathsf{SC}}}



\def \Pwd    {\mdef{\mathcal{PWD}}}
\def \Id    {\mdef{\mathcal{ID}}}
\def \pwd   {\mdef{\mathtt{pwd}}}
\def \salt   {\mdef{\mathtt{salt}}}
\def \msg   {\mdef{\mathsf{msg}}}
\def \id   {\mdef{\mathtt{Id}}}



%\def \RNU    {\mdef{\mathsf{RegisterNewUser}}}

\def \QSgin    {\mdef{\mathsf{RegisterQuery}}}

\def \Lgin    {\mdef{\mathsf{Login}}}
\def \QLgin    {\mdef{\mathsf{LoginQuery}}}
\def \Updt    {\mdef{\mathsf{Update}}}
\def \Act    {\mdef{\mathsf{Act}}}
\def \PwdUp    {\mdef{\mathsf{PwdUpdate}}}
\def \PBE    {\mdef{\mathsf{PBE}}}
\def \SE    {\mdef{\mathsf{SE}}}
\def \SH    {\mdef{\mathsf{SH}}}
\def \PCache    {\mdef{\mathsf{PCache}}}
\def \SDis  {\mdef{\mathsf{SmallDistance}}}
\def \TpDis  {\mdef{\mathsf{tpDist}}}
\def \TgDis  {\mdef{\mathsf{tgtDist}}}
\def \Hamm  {\mdef{\mathsf{Hamming}}}
\def \Edit  {\mdef{\mathsf{Edit}}}
\def \Dist  {\mdef{\mathtt{dist}}}

\def \Del {\mdef{\mathtt{Delete}}}
\def \Ins {\mdef{\mathtt{Insert}}}
\def \Chng {\mdef{\mathtt{Change}}}
\def \Trasn {\mdef{\mathtt{Transfer}}}

\def \Extr  {\mdef{\mathtt{Extract}}}
\def \EqPwdVec {\mdef{\mathtt{EqPwdVec}}}
\def \ValidLginQ {\mdef{\mathtt{ValidLginQuery}}}
\def \ToInt  {\mdef{\mathsf{ToInt}}}
\def \ToOrig  {\mdef{\mathsf{ToInt}^{-1}}}
\def \Pad  {\mdef{\mathtt{Pad}}}
\def \UPad  {\mdef{\mathtt{UnPad}}}
\def \Pail {\mdef{\mathtt{P}}}
\def \SS {\mdef{\mathtt{SS}}}
\def \PtoCMul {\mdef{\mathtt{PlainToCtxMul}}}
\def \Add {\mdef{\mathtt{Add}}}
\def \Sum {\mdef{\Sigma}}
\def \Sub {\mdef{\mathtt{Subtract}}}
\def \lcm {\mdef{\mathsf{lcm}}}
\def \Zs {\mdef{\mathbb{Z}^*}}
\newcommand\Mbb[1]{\mathbb{#1}}
%predicates macros
\def \HammOne {\mdef{\mathtt{HammDisOne}}}
\def \ArbHamm {\mdef{\mathtt{HammDis}}}
\def \Capslock {\mdef{\mathsf{CAPSLOCK}}}
\def \oneDel {\mdef{\mathtt{OneCharDel}}}
\def \oneIns {\mdef{\mathtt{OneCharIns}}}
\def \Sum   {\mdef{\mathtt{Sum}}}
\def \RanEnc   {\mdef{\mathsf{REnc}}}
\def \RanDec   {\mdef{\mathsf{RDec}}}

\def \Auth   {\mdef{\mathsf{Auth}}}


\def \Ham  {\mdef{\mathtt{Ham}}}
\def \ED  {\mdef{\mathtt{ED}}}
\def \ToLowerCase  {\mdef{\mathtt{ToLowerCase}}}

\def \apnd  {\mdef{\mathtt{Append}}}

\def \BboxP  {\mdef{\mbox{\fontsize{13.28}{21.6}\selectfont\( %
			\boxplus%
			\)} }}




\def \Init {\mdef{\mathsf{Initialization}}}
\def \Sgin {\mdef{\mathsf{RegisterNewUser}}}
\def \Lgin {\mdef{\mathsf{Login}}}
\def \EvType {\mdef{\mathsf{EventType}}}
\def \Register {\mdef{\mathsf{Register}}}
\def \Login {\mdef{\mathsf{Login}}}
\def \PwdUpd {\mdef{\mathsf{PwdUpdate}}}
\def \acpt {\mdef{\mathtt{acpt}}}
\def \rjct {\mdef{\mathtt{rjct}}}
\def \isReg {\mdef{\mathtt{isRegistered}}}
\def \PullPwd {\mdef{\mathtt{PullPwd}}}
\def \view {\mdef{\mathtt{view}}}

\def \Share   {\mdef{\mathsf{ShareGen}}}
\def \recover   {\mdef{\mathsf{SecretRecover}}}
\def \ValidShr   {\mdef{\mathsf{ValidShare}}}
\def \InterPol   {\mdef{\mathsf{InterPol}}}
\def \shrs   {\mdef{\mathsf{shares}}}
\def \Diffx   {\mdef{\mathtt{Diff}}}
\def \Experiment   {\mdef{\mathtt{Experiment}}}
\def \Equ   {\mdef{\mathtt{Equ}}}

\def \badRng   {\mdef{\mathtt{BadRange}}}

\newcommand{\InvertCase}{\mathsf{InvertCase}}

\def \iif   {\mdef{\mathtt{if~}}}
\def \eelse   {\mdef{\mathtt{else~}}}
\def \ffor   {\mdef{\mathtt{for~}}}
\def \bbreak   {\mdef{\mathtt{break~}}}
\def \ooutput   {\mdef{\mathtt{output~}}}
\def \ddiscard   {\mdef{\mathtt{discard~}}}
\def \ccompute   {\mdef{\mathtt{compute~}}}
\def \sset   {\mdef{\mathtt{set~}}}
\def \ssample   {\mdef{\mathtt{sample~}}}
\def \pparse   {\mdef{\mathtt{parse~}}}
\def \aand   {\mdef{\mathtt{~AND~}}}
\def \oor   {\mdef{\mathtt{~OR~}}}
\def \run   {\mdef{\mathtt{run~}}}
\def \rreturn   {\mdef{\mathtt{return~}}}
\def \condit   {\mdef{\mathtt{Cond}}}

\newcommand{\ldb}{\mathopen{\lbrack\!\lbrack}} 
\newcommand{\rdb}{\mathclose{\rbrack\!\rbrack}}




\renewcommand{\algorithmiccomment}[1]{\hfill$\triangleright${\color{purple} #1}}
\renewcommand{\algorithmicrequire}{\textbf{Input:}}
\renewcommand{\algorithmicensure}{\textbf{Output:}}
%---------------------------------------------------------------------------------------------------------------------------------------------------

%---------------------------------------------------------------------------------------------------------------------------------------------------

\newcommand\norm[1]{\left\lVert#1\right\rVert}
\newcommand{\PPr}[1]{\ensuremath{\mathbf{Pr}\left[#1\right]}}
\newcommand{\PPPr}[2]{\ensuremath{\underset{#1}{\mathbf{Pr}}\left[#2\right]}}
\newcommand{\Ex}[1]{\ensuremath{\mathbb{E}\left[#1\right]}}
\newcommand{\EEx}[2]{\ensuremath{\underset{#1}{\mathbb{E}}\left[#2\right]}}
\renewcommand{\O}[1]{\ensuremath{\mathcal{O}\left(#1\right)}}
\newcommand{\eps}{\epsilon}

\renewcommand{\labelenumi}{(\arabic{enumi})}
%\newcommand{\construction}[1]{\textbf{Construction} #1}
\renewcommand\labelitemii{\mbox{$\circ$}}
\renewcommand\labelitemiii{\mbox{\tiny$\spadesuit$}}
\renewcommand{\figurename}{Figure}

%%% Unified Formating
\newcommand{\mdef}[1]{{\ensuremath{#1}}\xspace}  % Math Def which can also be used in normal text.
\newcommand{\mydist}[1]{\mdef{\mathcal{#1}}}     % Distribution should use mathcal.
\newcommand{\myset}[1]{\mdef{\mathbb{#1}}}       % (Important) Sets should use mathbb.
\newcommand{\myalg}[1]{\mdef{\mathcal{#1}}}       % Algorithms should use mathtt.
\newcommand{\myfunc}[1]{\mdef{\mathsf{#1}}}      % Functions denoted with text (e.g. size()) should use mathsf.
\newcommand{\myvec}[1]{\mathbf{#1}}               % Vector
\def \ds {\displaystyle}                         % Shorthand for ``displaystyle''.
\DeclareMathOperator*{\argmin}{argmin}
\DeclareMathOperator*{\argmax}{argmax}
\DeclareMathOperator*{\polylog}{polylog}
\DeclareMathOperator*{\poly}{poly}

%%% Text notation
\newcommand{\superscript}[1]{\ensuremath{^{\mbox{\tiny{\textit{#1}}}}}\xspace}
\def \th {\superscript{th}}     % 'The i-th entry it a list...' --> i\th
\def \st {\superscript{st}}     % 'The 1-st entry it a list...' --> 1\st
\def \nd {\superscript{nd}}     % 'The 2-nd entry it a list...' --> 2\nd
\def \rd {\superscript{rd}}     % 'The 3-rd entry it a list...' --> 3\rd
\def \etal{{\it et~al.}}

%%% Common Sets & Symbols & Functions
\def \size     {\mdef{\myfunc{size}}}                % Function outputing vector of cardinalities of a given vector of sets.
\def \negl     {\mdef{\myfunc{negl}}}                % Negliglbe function
%\def \polylog  {\mdef{\myfunc{polylog}}}             % Polylogarithm function
\def \polyloglog  {\mdef{\myfunc{polyloglog}}}       % Any function that is polynomial in the logarithm of the logarithm of the argument
\newcommand{\abs}[1]{\mdef{\left|#1\right|}}         % Absolute value
\newcommand{\flr}[1]{\mdef{\left\lfloor#1\right\rfloor}}              % Absolute value
\newcommand{\ceil}[1]{\mdef{\left\lceil#1\right\rceil}}               % Absolute value
\newcommand{\rdim}[1]{\mdef{\dim\left(#1\right)}}                     % Dimension of a vector
\newcommand{\set}[1]{\mdef{\left\{#1\right\}}}                        % Absolute value
\newcommand{\E}[2][]{\mdef{\underset{#1}{\mathbb{E}}\left[#2\right]}} % Expected value

\newcommand{\ignore}[1]{}


%%%% pebbling complexity notions
\def \cc       {\mdef{\mathsf{cc}}}
\newcommand{\peb}{\Pi} % pebbling complexity
\newcommand{\Peb}{{\cal P}} % set of legal pebblings

%\newcommand{\parallel}{\|}

\newcommand{\ppeb}{\peb^{\parallel}} % parallel versions of def above
\newcommand{\pPeb}{\Peb^{\parallel}}

\newcommand{\Cspace}{s}
\newcommand{\Ctime}{t}
\newcommand{\Ccc}{{cc}}
\newcommand{\Cspacetime}{{st}}
\newcommand{\Cevery}{\{\Cspace,\Ctime,\Cspacetime,\Ccc\}}

%\def \pcc {\ppeb_\Ccc} % shortcut for Parallel Cumulative Complexity
\def \pcc {\cc} % shortcut for Parallel Cumulative Complexity
\def \sst {\peb_{\Cspacetime}} % shourtcut for Sequential Space Time


\newif\ifnotes\notestrue %set this to true if notes are visible and to false (next line) if they should be hidden
% \newif\ifnotes\notesfalse
\ifnotes
\newcommand{\samson}[1]{\textcolor{purple}{{\bf (Samson:} {#1}{\bf ) }} \marginpar{\tiny\bf
             \begin{minipage}[t]{0.5in}
               \raggedright S:
            \end{minipage}}}       
            
\newcommand{\jeremiah}[1]{\textcolor{red}{{\bf (Jeremiah:} {#1}{\bf ) }} \marginpar{\tiny\bf
             \begin{minipage}[t]{0.5in}
               \raggedright J:
            \end{minipage}}    }         
        
\newcommand{\hassan}[1]{\textcolor{blue}{{\bf (Hassan:} {#1}{\bf ) }} \marginpar{\tiny\bf
		\begin{minipage}[t]{0.5in}
			\raggedright H:
\end{minipage}}    }
                							
\else
\newcommand{\samson}[1]{}
\newcommand{\jeremiah}[1]{}
\fi

\hypersetup{
     colorlinks   = true,
     citecolor    = blue,
		 linkcolor		= red
}

\makeatletter
\renewcommand*{\@fnsymbol}[1]{\textcolor{mahogany}{\ensuremath{\ifcase#1\or *\or \dagger\or \ddagger\or
 \mathsection\or \triangledown\or \mathparagraph\or \|\or **\or \dagger\dagger
   \or \ddagger\ddagger \else\@ctrerr\fi}}}
\makeatother

\providecommand{\email}[1]{\href{mailto:#1}{\nolinkurl{#1}\xspace}}

\definecolor{electricpurple}{rgb}{0.75, 0.0, 1.0}
\definecolor{fluorescentpink}{rgb}{1.0, 0.08, 0.58}
\definecolor{mahogany}{rgb}{0.75, 0.25, 0.0}
\definecolor{darkblue}{rgb}{0.0, 0.0, 0.55}
\definecolor{darkpastelgreen}{rgb}{0.01, 0.75, 0.24}
\definecolor{darkgreen}{rgb}{0.0, 0.2, 0.13}
\definecolor{darkgoldenrod}{rgb}{0.72, 0.53, 0.04}
\definecolor{darkred}{rgb}{0.55, 0.0, 0.0}
\hypersetup{
     colorlinks   = true,
     citecolor    = electricpurple,
		 linkcolor		= blue
}